%Papierformat
\documentclass[12pt,a4paper]{scrartcl}
%Deutsche Silbentrennung
\usepackage[ngerman]{babel}
%Listen einrücken
\usepackage{enumitem}
%Deutsche Umlaute
\usepackage[utf8]{inputenc}
%Trennung von deutschen Umlauten
\usepackage[T1]{fontenc}
%Für Zitate
\usepackage[numbers,square]{natbib}
%Grafikpaket laden
\usepackage{graphicx}
%Grafik Floating einschränken
\usepackage{placeins}
%Referenzierung mit Name
\usepackage{titleref}
%Unterschritszeilen
\usepackage{tabularx}
%Verlinkung
\usepackage{hyperref}
%Urls sollen erkannt werden
\usepackage{url}
%Erm�glicht es von Schrift umflossene Bilder und Tabellen einzufügen
\usepackage{wrapfig}

%\makeatletter
%\newcommand\footnoteref[1]{\protected@xdef\@thefnmark{\ref{#1}}\@footnotemark}
%\makeatother​

%Dokumentbeginn
\begin{document}

	%Festlegung des Zitatstyles - Harvardmethode: Abkürzung Autor + Jahr
	\bibliographystyle{alphadin}
	\graphicspath{{android\_pages/graphics}}

\begin{titlepage}
	%Eine mbox wird verwendet um Text zusammenzuhalten
	%vspace erzeugte die in Klammern angegebenen Zeilenabstände
	%baselineskip setzt zeilenabstand
   	\mbox{}\vspace{5\baselineskip}\\
   	%Schriftart und Größe als Attribut
   	\rmfamily\huge
   	%Mittige Textausrichtung (\centerline für eine Zeile)
   	\centering
   	%Das Argument erscheint in Kapitälchen (small capitals).
	\textsc{App-Security in iOS und Android}
	%Umbruch bezogen auf die Höhe des Kleinbuchstaben x in diesem Element * Faktor
	\\[3ex]
   	Seminararbeit
   	\rmfamily\Large
   	\vspace{1\baselineskip}\\
   	%Externes einbinden einer Textdatei
   	\input{version.txt}\mbox{}
	\vspace{3\baselineskip}\\
	Hochschule f"ur angewandte Wissenschaften W"urzburg-Schweinfurt
   	\vspace{5\baselineskip}\\
   	\rmfamily\Large
   	David Artmann\\
   	\rmfamily\Large
   	Kristoffer Schneider
   	\vspace{1\baselineskip}\\
   	%Heutiges Datum
   	\today
\end{titlepage}

	%Inhaltsverzeichnis
	\tableofcontents
	\newpage
	%Falls nötig...
	%Literaturliste im Inhaltsverzeichnis anzeigen
	%\addcontentsline{toc}{section}{Literatur}
	
	%Abbildungsverzeichnis
	\listoffigures
	
	%Vorwort
	\newpage
	\section{Vorwort}
	Das Smartphone ist in unserer heutigen Welt nicht mehr wegzudenken. Es dient
	als Alltagshelfer mit vielen Funktionen. Es kann als Notizbuch, oder
	Terminkalender genutzt werden. Es ist eines der Hauptkommunikationsmittel, im
	gesprochenen, als auch geschriebenen Wort. Ebenso wird es auch zum
	Zeitvertreib, oder für die Navigation genutzt. Dabei begleitet es uns jeden
	Tag fast überall.
	Mehrere Studien haben offen gelegt, dass das Smartphone den Computer bzw. Laptop - zumindest bei der
	Generation unter 18 Jahren - schlägt.
	\footnote{http://www.bitkom.org/files/documents/BITKOM\_PK\_Kinder\_und\_Jugend\_3\_0.pdf}
	\footnote{http://www.mpfs.de/fileadmin/JIM-pdf13/JIMStudie2013.pdf}
	Das Smartphone hat in den letzten Jahren sich in den Alltag der meisten
	Menschen eingefügt (Statistik?!?!).  Es dient als Alltagshelfer, Notizbuch, 
	Terminkalender, Kommunikationsmittel und Zeitvertreib, und ist dabei fast überall dabei.\\
	Viele vergessen dabei, dass Smartphones mittlerweile die Leistung eines kleinen
	Computers haben und somit auch die selben Gefahren wie am PC Zuhause vorhanden sind.
	Kaum einer hat auf seinem PC kein Anti-Viren System installiert. Aber wer hat
	eines auf seinem Smartphone oder Tablet? Dabei hat man gerade auf diesen Geräten zum Teil 
	hoch sensible Daten gespeichert.\\
	Daher wollen wir im Folgenden auf sicherheitstechnische Aspekte des Android und
	iOS Betriebssystems eingehen und aufzeigen welche Hilfsmittel beide für
	Entwickler und Nutzer bereitstellen.

	
	%Das Android und iOS Betriebssystem 
	\newpage
	
\section{Das Android Betriebssystem}
	Das Unternehmen Android wurde 2003 von Andy Rubin gegründet und wurde 2005 von Google aufgekauft. Seitdem kümmert sich Google und das Android Open Source Project (AOSP) um die Weiterentwicklung des Systems. Zuletzt wurden die neuen Versionen jeweils von Google intern entwickelt und zum Release der Version der AOSP Community als Open-Source bereitgestellt.\\
	Aktuell ist Android, mit 55.6\% Marktanteil \cite{MobileOsStat}, das vorherrschende Betriebssystem für mobile Endger"ate in den USA.
	\\\\
	Basis für das Betriebssystem ist ein modifizierter Linux-Kernel und eine Java Virtual Machine (JVM). Bis einschliesslich Version 4.4 wurde hierfür die Dalivk Runtime und für alle neueren Versionen die Android Runtime (ART) verwendet. Jede App l"auft in einer eigenen Instanz der entsprechenden Runtime und damit in einer Sandbox.\newline
	Oberhalb der JVM sind die meisten Komponenten in Java implementiert.
	
	\begin{figure}[h]
		\centering
		\includegraphics[width=0.7\linewidth]{android_pages/graphics/architektur_android_.png}
		\caption{Die Android Architektur \protect\cite{Elenkov2014} }
		\label{fig:architektur_android}
	\end{figure}
	\newpage
	\section{Das iOS Betriebsystem}
	Die aus Cupertino in Kalifornien stammende US-Amerikanische Apple Corporation
	ist eine der größten Firmen der Welt und hatte einen Umsatz von 182 Milliarden
	US-Dollar im Geschäftsjahr 2014. Sie ist der Erfinder des mobilen
	Betriebssystems iOS, welches auf den firmeneigenen Geräten iPad, iPad mini,
	iPhone, iPod touch und dem Apple TV ab der zweiten Generation zum Einsatz
	kommt. Als Steve Jobs 1985 das Unternehmen verließ, gründete er
	kurze Zeit darauf die Firma NeXT, mit welcher er unter anderem das
	Betriebssystem NeXTStep entwickelte, welches auf dem UNIX ähnlichem
	Betriebssystem BSD\cite[S.12]{Tanenbaum2009} und dem Mach-2.5-Kernel
	\cite{MachProject2015} basiert. NeXT wurde 1996 von Apple aufgekauft und Jobs
	kehrte als CEO zu Apple zurück. Damit begann die Karriere des mobilen
	Betriebssystems iOS. Zuerst wurde NeXTStep als Portierung in Form von Mac OS X
	(später OS X) weiter entwickelt. Mac OS X ist wiederrum der Vorleger für das
	iPhone OS (später iOS), welches am 09. Januar 2007 mit dem damals neu
	erschienenen iPhone erstmals vorgestellt wurde. Im März 2015 hatte iOS in den USA einen
	Marktanteil von 36,5\% und 18,3\% an den insgesamt genutzten mobilen
	Betriebssystemen in Deutschland\cite{MobileOsStat}.
	
	\begin{figure}[h]
		\centering
		\includegraphics[width=0.5\linewidth]{ios/media/marketshare-cmp-201503.jpg}
		\caption{Marktanteil der mobilen Betriebssysteme
		\cite{MobileOsStat}}
		\label{fig:marcetshare}
	\end{figure}
	
	%Apple's law enforcement process guidelines
	\newpage
	\section{Apple's Law Enforcement Process Guidelines}
	Die Firma Apple schreibt auf ihrer
	Webseite\footnote{https://www.apple.com/privacy/government-information-requests/}:
	\begin{quote}
	Our commitment to customer privacy doesn't stop because of a government
	information request.
	\end{quote}
	Weiterhin wird beteuert:
	%TODO: edit this, because the content has changed!!!
	\begin{quote}
	In addition, Apple has never worked with any government agency from any country to create a back 
	door in any of our products or services.
	\end{quote}
	Apple beteuert hier, seine Verpflichtung zur Einhaltung der Privatssphäre des
	Kunden auch nicht einzustellen, wenn die Regierung um Auskunft genau dieser
	Daten bittet.\\
	Zusätzlich wird versichert, dass Apple niemals mit Regierungsbehörden
	jedweder Länder gearbeitet hat, um Trojaner oder andere Hintertüren in eines
	ihrer Produkte oder Dienstleistungen einzubauen.\\
	%TODO: is this really necessary??
	Eine Einhaltung dieser Richtlinien ist von Kundenseite natürlich gewünscht.
	Ob Apple hier wirklich Wort hält, werde ich auf den kommenden Seiten genauer
	beleuchten.

	
	%Grundlegender Aufbau einer Android App
	\newpage
		\section{Grundlegender Aufbau einer Android App}
	\begin{quote}
	Android apps are written in the Java programming language. The Android SDK tools compile your code - along with any data and resource files - into an APK: an \textit{Android package}, which is an archive file with an .apk suffix. One APK file contains all the contents of an Android app and is the file that Android-powered devices use to install the app.
	\end{quote}
	
	Eine App besteht im Kern aus zwei Teilen. Den eigentlichen Programmkomponenten und einer Manifest Datei.
	\\
	Als Programmkomponenten k"onnen unter anderem vorkommen:
	\begin{itemize}\itemsep0pt
		\item Activities - stellen die Benutzeroberfl"ache dar
		\item Services - kann im Hintergrund laufen, auch wenn die App minimiert ist
		\item Content Provider - stellt Daten für die eigene und evtl f"ur andere Apps zur Verf"ugung
		\item Broadcast Receiver - um Systemweite Benachrichtigungen zu empfangen (z.B dass ein Download beendet wurde)
	\end{itemize}
	In der Manifest-Datei werden Eigenschaften der App definiert. Darunter z"ahlen beispielsweisse:
	\begin{itemize}\itemsep0pt
		\item Name der App
		\item Ziel SDK-Versionen
		\item Versionsnummer
		\item optional eine UserId ( siehe \ref*{sec:BasisRechteSystem} )
		\item Permissions ( siehe \ref*{sec:SandBoxingNPermissions} )
	\end{itemize}
	Desweiteren muss jede App signiert werden. Das hierfür benötigte Zertifikat kann sich jeder selbst generieren und muss nicht durch einen Certification Authority ( CA ) beglaubigt werden. Dabei wird angeraten, dass ein Entwickler für all seine Apps dasselbe Zertifikat nutzt.
	
	%Sicherheitsarchitektur iOS
	\newpage
	\section{Sicherheitsarchitektur iOS}
	\subsection{Secure boot chain}
	Apple hat eine Kette aneinander gereihter, vom Vorgänger abhängiger Prozesse
	entwickelt, um den Startvorgang möglichst abzusichern und eine Manipulation
	der Low-Level Software auszuschließen. Dieses "`secure boot chain"'
	genannte Verfahren stellt sicher, dass iOS nur auf Geräten startet, welche
	auf diesem Wege erfolgreich validiert wurden. Dabei wird nach dem Start
	eines iOS Gerätes zuerst Code aus einem nur lesbaren Speicherbereich
	ausgeführt. Dieser hardware of trust genannte unveränderbare Code ist bei
	der Manufaktur der Chips eingebettet worden und somit implizit vertraulich. 
	Das Boot ROM enthält zusätzlich Apple's öffentlichen Schlüssel des Root 
	Zertifikats, welcher sicher stellt, dass der Low-Level-Bootloader von Apple 
	signiert ist, bevor er ausgeführt wird.
	
	%\begin{figure}[h]
		%\centering
 		%\includegraphics[width=0.3\linewidth]{ios/media/security-model.jpg}
        %\caption{Sicherheitsarchitektur Diagramm von iOS}
        %\label{fig:security-model}
    %\end{figure}\\
    %ref: Figure \ref{fig:security-model} shows the sec. arch.
    
    %\begin{wrapfigure}[Zeilen]{Position}[Ueberhang]{Breite}
	\begin{wrapfigure}[0]{r}[0.5cm]{6cm}
		\includegraphics[width=\linewidth]{ios/media/security-model.jpg}
		\caption{Sicherheitsmodel von iOS}
		\label{fig:security-model}
	\end{wrapfigure}
	\subsection{iOS Sicherheitsmodel}
	Apple integrierte vier Schichten der Sicherheit in iOS.\\
	
	\newpage 
	\section{Sicherheitsaspekte der Android-Architektur}

	Bereits durch die Architektur des Betriebssystems, insbesondere durch die restriktive Rechtevergabe und das Sandboxing, wird versucht ein möglichst sicheres System bereitzustellen.

	\subsection{Basis Rechtesystem}\label{sec:BasisRechteSystem}
	Von Linux wurde auch das Basis-Rechtesystem übernommen. Hierbei bekommt jede App eine eindeutige User-ID (UID) zugewiesen, welche im Normalfall zur Installationszeit zugeteilt wird. Jeder Nutzer, und somit auch jede App, arbeitet grundsätzlich erst einmal nur innerhalb der ihm zugewiesenen virtuellen Maschine und dem damit verbundenen Dateisystem.\\\\
	Da es dennoch in vielen Fällen nötig ist Daten zwischen verschiedenen Apps auszutauschen, gibt es mehrer Möglichkeiten dies zu tun. Die üblichen Wege wären Intends oder SharedPreferences. Zusätzlich gibt es noch die Möglichkeit mehreren Apps dieselbe UID zuweisen zu lassen. Dies ist allerdings nur möglich wenn die entsprechenden Applikationen mit dem selben Zertifikat signiert wurden und in deren Manifest Datei eine gemeinsame UID festgelegt wurde.
	Durch dieses Rechtesystem wird versucht sicherzustellen, dass kein Nutzerprogramm als \textit{root} ausgeführt wird.
	
	\subsection{Sandboxing und Permissions} \label{sec:SandBoxingNPermissions}
	
	%Geheime Dienste
	\newpage
	\input{ios/pages/3_proprietary_and_opensource/geheime_dienste.tex}
	
	\newpage

\end{document}
