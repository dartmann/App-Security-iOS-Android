%Papierformat
\documentclass[12pt,a4paper]{scrartcl}
%deutsche Silbentrennung
\usepackage[ngerman]{babel}
%Listen einrücken
\usepackage{enumitem}
%deutsche Umlaute
\usepackage[utf8]{inputenc}
%Trennung von deutschen Umlauten
\usepackage[T1]{fontenc}
%für Zitate
\usepackage[numbers,square]{natbib}
%Grafikpaket laden
\usepackage{graphicx}
%Grafik Floating einschränken
\usepackage{placeins}
%Referenzierung mit Name
\usepackage{titleref}
%Unterschritszeilen
\usepackage{tabularx}
%Verlinkung
\usepackage{hyperref}

%\makeatletter
%\newcommand\footnoteref[1]{\protected@xdef\@thefnmark{\ref{#1}}\@footnotemark}
%\makeatother​

%Dokumentbeginn
\begin{document}

	%Festlegung des Zitatstyles - Harvardmethode: Abkuerzung Autor + Jahr
	\bibliographystyle{alphadin}

\begin{titlepage}
	%Eine mbox wird verwendet um Text zusammenzuhalten
	%vspace erzeugte die in Klammern angegebenen Zeilenabstände
	%baselineskip setzt zeilenabstand
   	\mbox{}\vspace{5\baselineskip}\\
   	%Schriftart und Größe als Attribut
   	\rmfamily\huge
   	%Mittige Textausrichtung (\centerline für eine Zeile)
   	\centering
   	%Das Argument erscheint in Kapitälchen (small capitals).
	\textsc{Security in iOS}
	%Umbruch bezogen auf die Höhe des Kleinbuchstaben x in diesem Element * Faktor
	\\[3ex]
   	Seminararbeit
   	\rmfamily\Large
   	\vspace{1\baselineskip}\\
   	%Externes einbinden einer Textdatei
   	\input{version.txt}\mbox{}
	\vspace{3\baselineskip}\\
	Hochschule f"ur angewandte Wissenschaften W"urzburg-Schweinfurt
   	\vspace{5\baselineskip}\\
   	\rmfamily\Large
   	David Artmann
   	\vspace{1\baselineskip}\\
   	%Heutiges Datum
   	\today
\end{titlepage}

	%Inhaltsverzeichnis
	\tableofcontents
	\newpage
	%Falls nötig...
	%Literaturliste im Inhaltsverzeichnis anzeigen
	%\addcontentsline{toc}{section}{Literatur}
	
	%Abbildungsverzeichnis
	\listoffigures
	\newpage
	
	\section{Vorwort}
	Die aus Cupertino in Kalifornien stammende US-Amerikanische Apple Corporation
	ist eine der gr"o"ssten Firmen der Welt und hatte einen Umsatz von 182 Mrd.
	USD im Gesch"aftsjahr 2014. Sie ist ebenfalls der Erfinder des mobilen
	Betriebssystems iOS, welches auf den firmeneigenen Ger"aten iPad, iPad mini,
	iPhone, iPod touch und dem Apple TV ab der zweiten Generation zum Einsatz kommt.
	Der Kern des Betriebssystems basiert auf dem freien UNIX Betriebssystem Darwin,
	das auch als Vorlage f"ur das Betriebssystem OS X genutzt wurde.\\
	Im Februar 2015 hatte iOS in den USA einen Marktanteil von 38,8 Prozent und 17,4 Prozent in Deutschland.
	\footnote{http://www.kantarworldpanel.com/global/smartphone-os-market-share/}
	Es existieren mehr als einhundert Millionen iPhones auf der ganzen Welt. Was
	also w"urde es bedeuten, wenn Sicherheitsl"ucken in diesen zu finden und
	auszunutzen w"aren.
	Mit dieser Arbeit m"ochte ich besonders auf Sicherheitstechnische Problematiken
	dieses Betriebssystems eingehen und zeigen, dass ein Propriet"ares
	Betriebssystem viele Vorteile hat, aber ebenso auch Nachteile besitzt die es nicht zu verachten gilt. 
	Ebenso werde ich versuchen mit praktischen Beispielen zu zeigen, dass iOS -
	obwohl propriet"ar - dennoch angreifbar ist und L"ucken aufweist.
	\newpage
	\section{Apple's Law Enforcement Process Guidelines}
	Die Firma Apple schreibt auf ihrer Webseite:
	\begin{quote}
	Our commitment to customer privacy doesn't stop because of a government
	information request.
	\end{quote}
	Weiterhin wird beteuert: 
	\begin{quote}
	In addition, Apple has never worked with any government agency from any country to create a back door in any of our products or services.
	\end{quote}
	\textsl{Wer das liest ist doof}
	
	\newpage
	\section{Geheime Dienste}
	Apple kann SMS, Fotos, Videos, Kontakte, Musik, Aufnahmen und Anruferhistorie
	aus passcode gesch"utzten Ger"aten auslesen. M"oglich machen dies nicht
	dokumentierte Dienste, welche auf jedem Ger"at mit iOS installiert sind. In
	diesem Kapitel will ich auf diese Dienste eingehen und deren genauen
	Einsatzzweck erl"autern.
	\subsection{lockdownd - remote access}
	Der Dienst \textsl{lockdownd} erm"olicht den Zugriff auf ein iOS Ger"at
	per TCP oder USB-Anschluss auf Port 62078.
	\newpage

\end{document}
