%Papierformat, mit Koma-Script Dokumentenklasse (Europäisches Design)
\documentclass[12pt,a4paper]{scrartcl}
%Deutsche Silbentrennung
\usepackage[ngerman]{babel}
%Listen einrücken
\usepackage{enumitem}
%Deutsche Umlaute
\usepackage[utf8]{inputenc}
%Trennung von deutschen Umlauten
\usepackage[T1]{fontenc}
%Bibtex für Zitate,
% see: https://en.wikibooks.org/wiki/LaTeX/Bibliography_Management#Customization
\usepackage[]{natbib}
%Grafikpaket laden
\usepackage{graphicx}
%Grafik Floating einschränken
\usepackage{placeins}
%Referenzierung mit Name
\usepackage{titleref}
%Unterschritszeilen
\usepackage{tabularx}
%Verlinkung
\usepackage[colorlinks=true,linkcolor=blue]{hyperref}
%Urls sollen erkannt werden
\usepackage{url}
%Ermöglicht es von Schrift umflossene Bilder und Tabellen einzufügen
\usepackage{wrapfig}
%Zitieren
%\usepackage{cite}
%Umbennen der Bibliography Angabe in Literatur
\renewcommand{\bibname}{Literatur}


%\makeatletter
%\newcommand\footnoteref[1]{\protected@xdef\@thefnmark{\ref{#1}}\@footnotemark}
%\makeatother​

%Dokumentbeginn
\begin{document}
	
	\input{general/titlepage.tex}

	%Inhaltsverzeichnis
	\tableofcontents
	\newpage
	
	%Abbildungsverzeichnis
	\listoffigures
	
	%Vorwort
	\newpage
	\section{Vorwort}
	Das Smartphone ist in unserer heutigen Welt nicht mehr wegzudenken. Es dient
	als Alltagshelfer mit vielen Funktionen. Es kann als Notizbuch, oder
	Terminkalender genutzt werden. Es ist eines der Hauptkommunikationsmittel, im
	gesprochenen, als auch geschriebenen Wort. Ebenso wird es auch zum
	Zeitvertreib, oder für die Navigation genutzt. Dabei begleitet es uns jeden
	Tag fast überall.
	Mehrere Studien haben offen gelegt, dass das Smartphone den Computer bzw. Laptop - zumindest bei der
	Generation unter 18 Jahren - schlägt.
	\footnote{http://www.bitkom.org/files/documents/BITKOM\_PK\_Kinder\_und\_Jugend\_3\_0.pdf}
	\footnote{http://www.mpfs.de/fileadmin/JIM-pdf13/JIMStudie2013.pdf}
	Das Smartphone hat in den letzten Jahren sich in den Alltag der meisten
	Menschen eingefügt (Statistik?!?!).  Es dient als Alltagshelfer, Notizbuch, 
	Terminkalender, Kommunikationsmittel und Zeitvertreib, und ist dabei fast überall dabei.\\
	Viele vergessen dabei, dass Smartphones mittlerweile die Leistung eines kleinen
	Computers haben und somit auch die selben Gefahren wie am PC Zuhause vorhanden sind.
	Kaum einer hat auf seinem PC kein Anti-Viren System installiert. Aber wer hat
	eines auf seinem Smartphone oder Tablet? Dabei hat man gerade auf diesen Geräten zum Teil 
	hoch sensible Daten gespeichert.\\
	Daher wollen wir im Folgenden auf sicherheitstechnische Aspekte des Android und
	iOS Betriebssystems eingehen und aufzeigen welche Hilfsmittel beide für
	Entwickler und Nutzer bereitstellen.

	
	%Das Android Betriebssystem 
	\newpage
	
\section{Das Android Betriebssystem}
	Das Unternehmen Android wurde 2003 von Andy Rubin gegründet und wurde 2005 von Google aufgekauft. Seitdem kümmert sich Google und das Android Open Source Project (AOSP) um die Weiterentwicklung des Systems. Zuletzt wurden die neuen Versionen jeweils von Google intern entwickelt und zum Release der Version der AOSP Community als Open-Source bereitgestellt.\\
	Aktuell ist Android, mit 55.6\% Marktanteil \cite{MobileOsStat}, das vorherrschende Betriebssystem für mobile Endger"ate in den USA.
	\\\\
	Basis für das Betriebssystem ist ein modifizierter Linux-Kernel und eine Java Virtual Machine (JVM). Bis einschliesslich Version 4.4 wurde hierfür die Dalivk Runtime und für alle neueren Versionen die Android Runtime (ART) verwendet. Jede App l"auft in einer eigenen Instanz der entsprechenden Runtime und damit in einer Sandbox.\newline
	Oberhalb der JVM sind die meisten Komponenten in Java implementiert.
	
	\begin{figure}[h]
		\centering
		\includegraphics[width=0.7\linewidth]{android_pages/graphics/architektur_android_.png}
		\caption{Die Android Architektur \protect\cite{Elenkov2014} }
		\label{fig:architektur_android}
	\end{figure}
	
	%Das iOS Betriebssystem
	\newpage
	\section{Das iOS Betriebsystem}
	Die aus Cupertino in Kalifornien stammende US-Amerikanische Apple Corporation
	ist eine der größten Firmen der Welt und hatte einen Umsatz von 182 Milliarden
	US-Dollar im Geschäftsjahr 2014. Sie ist der Erfinder des mobilen
	Betriebssystems iOS, welches auf den firmeneigenen Geräten iPad, iPad mini,
	iPhone, iPod touch und dem Apple TV ab der zweiten Generation zum Einsatz
	kommt. Als Steve Jobs 1985 das Unternehmen verließ, gründete er
	kurze Zeit darauf die Firma NeXT, mit welcher er unter anderem das
	Betriebssystem NeXTStep entwickelte, welches auf dem UNIX ähnlichem
	Betriebssystem BSD\cite[S.12]{Tanenbaum2009} und dem Mach-2.5-Kernel
	\cite{MachProject2015} basiert. NeXT wurde 1996 von Apple aufgekauft und Jobs
	kehrte als CEO zu Apple zurück. Damit begann die Karriere des mobilen
	Betriebssystems iOS. Zuerst wurde NeXTStep als Portierung in Form von Mac OS X
	(später OS X) weiter entwickelt. Mac OS X ist wiederrum der Vorleger für das
	iPhone OS (später iOS), welches am 09. Januar 2007 mit dem damals neu
	erschienenen iPhone erstmals vorgestellt wurde. Im März 2015 hatte iOS in den USA einen
	Marktanteil von 36,5\% und 18,3\% an den insgesamt genutzten mobilen
	Betriebssystemen in Deutschland\cite{MobileOsStat}.
	
	\begin{figure}[h]
		\centering
		\includegraphics[width=0.5\linewidth]{ios/media/marketshare-cmp-201503.jpg}
		\caption{Marktanteil der mobilen Betriebssysteme
		\cite{MobileOsStat}}
		\label{fig:marcetshare}
	\end{figure}
	%Apple's law enforcement process guidelines
	\section{Apple's Law Enforcement Process Guidelines}
	Die Firma Apple schreibt auf ihrer
	Webseite\footnote{https://www.apple.com/privacy/government-information-requests/}:
	\begin{quote}
	Our commitment to customer privacy doesn't stop because of a government
	information request.
	\end{quote}
	Weiterhin wird beteuert:
	%TODO: edit this, because the content has changed!!!
	\begin{quote}
	In addition, Apple has never worked with any government agency from any country to create a back 
	door in any of our products or services.
	\end{quote}
	Apple beteuert hier, seine Verpflichtung zur Einhaltung der Privatssphäre des
	Kunden auch nicht einzustellen, wenn die Regierung um Auskunft genau dieser
	Daten bittet.\\
	Zusätzlich wird versichert, dass Apple niemals mit Regierungsbehörden
	jedweder Länder gearbeitet hat, um Trojaner oder andere Hintertüren in eines
	ihrer Produkte oder Dienstleistungen einzubauen.\\
	%TODO: is this really necessary??
	Eine Einhaltung dieser Richtlinien ist von Kundenseite natürlich gewünscht.
	Ob Apple hier wirklich Wort hält, werde ich auf den kommenden Seiten genauer
	beleuchten.

	
	%Grundlegender Aufbau einer Android App
	%\newpage
		\section{Grundlegender Aufbau einer Android App}
	\begin{quote}
	Android apps are written in the Java programming language. The Android SDK tools compile your code - along with any data and resource files - into an APK: an \textit{Android package}, which is an archive file with an .apk suffix. One APK file contains all the contents of an Android app and is the file that Android-powered devices use to install the app.
	\end{quote}
	
	Eine App besteht im Kern aus zwei Teilen. Den eigentlichen Programmkomponenten und einer Manifest Datei.
	\\
	Als Programmkomponenten k"onnen unter anderem vorkommen:
	\begin{itemize}\itemsep0pt
		\item Activities - stellen die Benutzeroberfl"ache dar
		\item Services - kann im Hintergrund laufen, auch wenn die App minimiert ist
		\item Content Provider - stellt Daten für die eigene und evtl f"ur andere Apps zur Verf"ugung
		\item Broadcast Receiver - um Systemweite Benachrichtigungen zu empfangen (z.B dass ein Download beendet wurde)
	\end{itemize}
	In der Manifest-Datei werden Eigenschaften der App definiert. Darunter z"ahlen beispielsweisse:
	\begin{itemize}\itemsep0pt
		\item Name der App
		\item Ziel SDK-Versionen
		\item Versionsnummer
		\item optional eine UserId ( siehe \ref*{sec:BasisRechteSystem} )
		\item Permissions ( siehe \ref*{sec:SandBoxingNPermissions} )
	\end{itemize}
	Desweiteren muss jede App signiert werden. Das hierfür benötigte Zertifikat kann sich jeder selbst generieren und muss nicht durch einen Certification Authority ( CA ) beglaubigt werden. Dabei wird angeraten, dass ein Entwickler für all seine Apps dasselbe Zertifikat nutzt.
	
	%Sicherheitsarchitektur iOS
	\newpage
	\section{Systemsicherheit unter iOS}
	Eine Reihe von aneinander gereihten und von einander abhängigen Prozessen
	trägt maßgeblich zur Systemsicherheit bei. Dies berücksichtigt vor allem den
	Startvorgang, die Software Updates - auch von Drittanbietern - und den Secure
	Enclave. Dies stellt sicher, dass alle Kernkomponenten, ob Hard- oder
	Software, möglichst gefeit vor Angriffen sind, ohne dabei die
	Nutzerfreundlichkeit zu beeinflussen. Wenn dabei ein dieser Schritte invalide
	ist, unterbricht der Startvorgang und das Gerät wird in den Recovery Modus
	versetzt. Wenn der Boot-ROM nicht geladen werden kann, wird der DFU (Device
	Firmware Upgrade) Modus betreten. Nachfolgend werden die an diesem Prozess
	beteiligten Komponenten detailliert vorgestellt.
	%\begin{wrapfigure}[Zeilen]{Position}[Ueberhang]{Breite}
	%\begin{wrapfigure}[0]{r}[0.5cm]{6cm}
	\begin{wrapfigure}[3]{r}{5cm}
		\includegraphics[width=\linewidth]{ios/media/security-model.jpg}
		\caption{Sicherheitsmodel von iOS}
		\label{fig:security-model}
	\end{wrapfigure}

	\subsection{Secure boot chain}
		%TODO: design in a nicer way, when this subsection is finished
		Dieses Verfahren stellt eine Manipulation der \\
		Low-Level Software sicher. Nur iOS Geräte, \\
		die eine erfolgreiche Validierung dieser\\ 
		Vertrauenskette bestanden haben, starten\\ 
		ordnungsgemäß. Dabei wird nach dem Start\\
		eines iOS Gerätes zuerst Code aus einem nur lesbaren\\
		Speicherbereich ausgeführt. Dieser \textit{hardware of trust}\\ 
		genannte und unveränderbare Code ist bei der\\
		Manufaktur der Chips eingebettet worden und somit\\
		implizit vertraulich. Das Boot ROM enthält zusätzlich\\
		den öffentlichen Schlüssel der Wurzel\\ 
		Zertifizierungsstelle von Apple,\\
		welcher eine Signierung des Low-Level-Bootloaders\\
		durch Apple sicher stellt, bevor er ausgeführt wird.\\ 
		Dies ist der erste Schritt in der "`chain of trust"',\\ 
		in welcher jeder Teilnehmer sicher stellt,\\ 
		dass der darauf	folgende von Apple signiert	ist.\\
		Nach erfolgreicher Abarbeitung aller Aufgaben des LLB,\\
		überprüft und startet dieser iBoot, den nächsten\\
		Bootloader, welcher wiederrum das selbe Prozedere mit\\
		dem iOS Kernel startet.\\
		Bei Geräten mit Mobilfunk Zugang führt das Basisband\\
		Untersystem seinen eigenen ähnlichen Prozess mit\\
		signierter Software und verifizierten Schlüsseln\\
		vom Basisband Prozessor durch.\\
		
		%\begin{figure}[h]
			%\centering
	 		%\includegraphics[width=0.3\linewidth]{ios/media/security-model.jpg}
	        %\caption{Sicherheitsarchitektur Diagramm von iOS}
	        %\label{fig:security-model}
	    %\end{figure}\\
	    %ref: Figure \ref{fig:security-model} shows the sec. arch.
	    
	    %\begin{wrapfigure}[Zeilen]{Position}[Ueberhang]{Breite}
		
	\subsection{Authorisierung von System Software}
		Dieser Prozess soll einen Downgrade auf eine ältere Version von iOS
		verhindern. Der im folgenden Kapitel besprochene Secure Enclave Co-Prozessor
		nutzt diese Technik für Integritätsprüfung seiner Software ebenfalls. Im Falle
		eines iOS Updates wird entweder von iTunes oder vom Gerät selbst einer der
		Apple Installations Authorisations Server kontaktiert. Diesem wird eine Liste
		von verschlüsselten Messungen pro Komponente, welche am Update
		beteiligt ist, gesandt. Zusätzlich wird ein	zufälliger Anti-Replay Wert und
		die eindeutige ID (ECID) des Gerätes verschickt. Diese Auflistung von gegen
		Versionen, für die ein Update genehmigt ist geprüft. Bei Übereinstimmung wird
		die ECID zu den Messungen hinzugefügt und das Ergebnis signiert, was einer
		Personalisierung der Daten gleicht.	Anschließend werden alle signierten Daten
		zum Gerät geschickt. Die sichere Startkette von Prozessen verifiziert die
		Signatur der empfangenen Daten auf den Absender Apple und ob die Prüfsumme der
		Signatur dem entspricht, was das lokale Ergebnis von verschlüsselten Messungen
		und ECID ergibt.\\
		Mit diesen Schritten wird eine Authorisierung nur für bestimmte Geräte sicher
		gestellt. Außerdem verhindert der Anti-Replay Wert ein mitschneiden der Server
		Antwort und das Verwenden der Daten bei anderen Geräten oder ein manipulieren
		der Daten.
	\subsection{Secure Enclave}
		%TODO: get more details for ensuring the facts in this chapter
		Dieser Co-Prozessor kommt in Geräten mit A7 oder jüngeren A-Serien Prozessoren
		vor. Er verwendet seinen eigenen sicheren Startvorgang, ist separiert vom
		Applikations Prozessor und verwendet verschlüsselten Speicher, sowie einen
		Hardware Zufallszahlen Generator. Die Technologie basiert auf ARM's
		TrustZone\footnote{http://www.arm.com/products/processors/technologies/trustzone/index.php}
		und wurde von Apple für die eigenen Ansprüche angepasst. Der Kernel dieser
		Einheit basiert auf der L4-Familie mit leichten Modifikationen. Die auf
		Interrupts basierende Kommunikation zwischen dem Applikationsprozessor und dem
		Secure Enclave (SE) läuft über einem nur den beiden zur Verfügung stehenden
		Speicherbereich. Der SE ist verantwortlich für das Schlüssel Management der
		Datenverschlüsselung und stellt die Integrität dieser sicher, auch wenn der
		Kernel des iOS Systems kompromitiert ist. Beim Herstellungsprozess erhält
		jeder SE eine einzigartige ID (UID), auf welche nur er zugreifen kann und
		welche auch Apple selbst nicht bekannt ist.	Mit dieser UID wird beim
		Systemstart der Speicherbereich des SE zusammen mit einem einmaligen Schlüssel
		verschlüsselt. Zusätzlich werden jegliche vom SE in den Speicher geschriebene
		Daten mit der UID und einem Anti-Replay Zufallswert verschlüsselt. Eine der
		Hauptaufgaben des Secure Enclave ist die verarbeitung der Fingerabdruck-Daten
		der Touch ID; im Detail vorgestellt im folgenden letzten Kapitel der
		Sicherheitskomponenten. Alle Kommunikation zwischen Touch ID und SE wird über
		einen seriellen Bus abgearbeitet. Der Applikationsprozessor leitet
		die Fingerabdrucksdaten an den SE weiter kann diese aber aufgrund einer
		Verschlüsselung der Daten mit einem Sitzungsschlüssel nicht lesen. Dieser
		Session Key wurde durch den für Touch ID und SE bereit gestellten
		öffentlichen Schlüssel erzeugt. Der Austausch des Sitzungsschlüssels wird
		durch AES Key Wrapping realisiert. Dabei erzeugen beide Seiten einen
		zufälligen Schlüssel, welche den Sitzungsschlüssel bilden. Zum verschlüsselten
		Transport wird AES-CCM genutzt.
	\subsection{Touch ID}
		\cite{Zdziarski2012}
	
	%\newpage 
	\section{Sicherheitsaspekte der Android-Architektur}

	Bereits durch die Architektur des Betriebssystems, insbesondere durch die restriktive Rechtevergabe und das Sandboxing, wird versucht ein möglichst sicheres System bereitzustellen.

	\subsection{Basis Rechtesystem}\label{sec:BasisRechteSystem}
	Von Linux wurde auch das Basis-Rechtesystem übernommen. Hierbei bekommt jede App eine eindeutige User-ID (UID) zugewiesen, welche im Normalfall zur Installationszeit zugeteilt wird. Jeder Nutzer, und somit auch jede App, arbeitet grundsätzlich erst einmal nur innerhalb der ihm zugewiesenen virtuellen Maschine und dem damit verbundenen Dateisystem.\\\\
	Da es dennoch in vielen Fällen nötig ist Daten zwischen verschiedenen Apps auszutauschen, gibt es mehrer Möglichkeiten dies zu tun. Die üblichen Wege wären Intends oder SharedPreferences. Zusätzlich gibt es noch die Möglichkeit mehreren Apps dieselbe UID zuweisen zu lassen. Dies ist allerdings nur möglich wenn die entsprechenden Applikationen mit dem selben Zertifikat signiert wurden und in deren Manifest Datei eine gemeinsame UID festgelegt wurde.
	Durch dieses Rechtesystem wird versucht sicherzustellen, dass kein Nutzerprogramm als \textit{root} ausgeführt wird.
	
	\subsection{Sandboxing und Permissions} \label{sec:SandBoxingNPermissions}
	
	%Geheime Dienste
	\newpage
	\input{ios/pages/3_proprietary_and_opensource/geheime_dienste.tex}


	%Referenzen zum Ende
	%Festlegung des Zitatstyles - Harvardmethode: Abkürzung Autor + Jahr
	\bibliographystyle{plain}
	%bibtex referenzen
	\newpage
	\bibliography{general/bibtex/bibtex.bib}
	%Bibliography im Inhaltsverzeichnis anzeigen(muss unterhalb von bibliography
	% sein)
	\addcontentsline{toc}{section}{Literatur}
\end{document}
