%Papierformat
\documentclass[12pt,a4paper]{scrartcl}
%deutsche Silbentrennung
\usepackage[ngerman]{babel}
%Listen einrücken
\usepackage{enumitem}
%deutsche Umlaute
\usepackage[ansinew]{inputenc}
%Trennung von deutschen Umlauten
\usepackage[T1]{fontenc}
%für Zitate
\usepackage[numbers,square]{natbib}
%Grafikpaket laden
\usepackage{graphicx}
%Grafik Floating einschränken
\usepackage{placeins}
%Referenzierung mit Name
\usepackage{titleref}
%Unterschritszeilen
\usepackage{tabularx}
%Verlinkung
\usepackage{hyperref}
%urls sollen erkannt werden
\usepackage{url}

%\makeatletter
%\newcommand\footnoteref[1]{\protected@xdef\@thefnmark{\ref{#1}}\@footnotemark}
%\makeatother​

%Dokumentbeginn
\begin{document}

	%Festlegung des Zitatstyles - Harvardmethode: Abkuerzung Autor + Jahr
	\bibliographystyle{alphadin}
	\graphicspath{{android\_pages/graphics}}

\begin{titlepage}
	%Eine mbox wird verwendet um Text zusammenzuhalten
	%vspace erzeugte die in Klammern angegebenen Zeilenabstände
	%baselineskip setzt zeilenabstand
   	\mbox{}\vspace{5\baselineskip}\\
   	%Schriftart und Größe als Attribut
   	\rmfamily\huge
   	%Mittige Textausrichtung (\centerline für eine Zeile)
   	\centering
   	%Das Argument erscheint in Kapitälchen (small capitals).
	\textsc{App-Security in iOS und Android}
	%Umbruch bezogen auf die Höhe des Kleinbuchstaben x in diesem Element * Faktor
	\\[3ex]
   	Seminararbeit
   	\rmfamily\Large
   	\vspace{1\baselineskip}\\
   	%Externes einbinden einer Textdatei
   	\input{version.txt}\mbox{}
	\vspace{3\baselineskip}\\
	Hochschule f"ur angewandte Wissenschaften W"urzburg-Schweinfurt
   	\vspace{5\baselineskip}\\
   	\rmfamily\Large
   	David Artmann\\
   	\rmfamily\Large
   	Kristoffer Schneider
   	\vspace{1\baselineskip}\\
   	%Heutiges Datum
   	\today
\end{titlepage}

	%Inhaltsverzeichnis
	\tableofcontents
	\newpage
	%Falls nötig...
	%Literaturliste im Inhaltsverzeichnis anzeigen
	%\addcontentsline{toc}{section}{Literatur}
	
	%Abbildungsverzeichnis
	\listoffigures
	
	%01.Vorwort
	\newpage
	\section{Vorwort}
	Die aus Cupertino in Kalifornien stammende US-Amerikanische Apple Corporation
	ist eine der größten Firmen der Welt und hatte einen Umsatz von 182 Mrd.
	USD im Geschäftsjahr 2014. Sie ist ebenfalls der Erfinder des mobilen
	Betriebssystems iOS, welches auf den firmeneigenen Geräten iPad, iPad mini,
	iPhone, iPod touch und dem Apple TV ab der zweiten Generation zum Einsatz kommt.
	Der Kern des Betriebssystems basiert auf dem freien UNIX Betriebssystem Darwin,
	das auch als Vorlage für das Betriebssystem OS X genutzt wurde.\\
	Im Februar 2015 hatte iOS in den USA einen Marktanteil von 38,8\% und 17,4\%
	in Deutschland.
	\footnote{http://www.kantarworldpanel.com/global/smartphone-os-market-share/}
	Es existieren mehr als einhundert Millionen iPhones auf der ganzen Welt.\\
	Das Smartphone ob iOS, Android oder Windows Phone als Betriebsystem, ist in
	unserer heutigen Welt nicht mehr wegzudenken und mehrere Studien haben offen
	gelegt, dass das Smartphone den Computer bzw. Laptop - zumindest bei der
	Generation unter 18 Jahren - schlägt.
	\footnote{http://www.bitkom.org/files/documents/BITKOM\_PK\_Kinder\_und\_Jugend\_3\_0.pdf}
	\footnote{http://www.mpfs.de/fileadmin/JIM-pdf13/JIMStudie2013.pdf}\\
	Welche Auswirkung hätte es, wenn Sicherheitslücken auf diesen zu finden wären,
	oder schlimmer noch, diese völlig ohne das Wissen des Nutzers ausgenutzt werden
	könnten? Mit dieser Arbeit möchte ich besonders auf Sicherheitstechnische
	Problematiken dieses Betriebssystems eingehen und zeigen, dass ein Proprietäres
	Betriebssystem viele Vorteile hat, aber ebenso auch Nachteile besitzt die es nicht zu verachten gilt. 
	Ebenso werde ich versuchen mit praktischen Beispielen zu zeigen, dass iOS -
	obwohl proprietär - dennoch angreifbar ist und Lücken aufweist.
	
	%Das Android Betriebssystem 
	\newpage
	
\section{Das Android Betriebssystem}
	Das Unternehmen Android wurde 2003 von Andy Rubin gegründet und wurde 2005 von Google aufgekauft. Seitdem kümmert sich Google und das Android Open Source Project (AOSP) um die Weiterentwicklung des Systems. Zuletzt wurden die neuen Versionen jeweils von Google intern entwickelt und zum Release der Version der AOSP Community als Open-Source bereitgestellt.\\
	Aktuell ist Android, mit 55.6\% Marktanteil \cite{MobileOsStat}, das vorherrschende Betriebssystem für mobile Endger"ate in den USA.
	\\\\
	Basis für das Betriebssystem ist ein modifizierter Linux-Kernel und eine Java Virtual Machine (JVM). Bis einschliesslich Version 4.4 wurde hierfür die Dalivk Runtime und für alle neueren Versionen die Android Runtime (ART) verwendet. Jede App l"auft in einer eigenen Instanz der entsprechenden Runtime und damit in einer Sandbox.\newline
	Oberhalb der JVM sind die meisten Komponenten in Java implementiert.
	
	\begin{figure}[h]
		\centering
		\includegraphics[width=0.7\linewidth]{android_pages/graphics/architektur_android_.png}
		\caption{Die Android Architektur \protect\cite{Elenkov2014} }
		\label{fig:architektur_android}
	\end{figure}
	
	%02.Apple's law enforcement process guidelines
	\newpage
	\input{ios_pages/02_apple_law_enforcement.tex}
	
	%Grundlegender Aufbau einer Android App
	\newpage
		\section{Grundlegender Aufbau einer Android App}
	\begin{quote}
	Android apps are written in the Java programming language. The Android SDK tools compile your code - along with any data and resource files - into an APK: an \textit{Android package}, which is an archive file with an .apk suffix. One APK file contains all the contents of an Android app and is the file that Android-powered devices use to install the app.
	\end{quote}
	
	Eine App besteht im Kern aus zwei Teilen. Den eigentlichen Programmkomponenten und einer Manifest Datei.
	\\
	Als Programmkomponenten k"onnen unter anderem vorkommen:
	\begin{itemize}\itemsep0pt
		\item Activities - stellen die Benutzeroberfl"ache dar
		\item Services - kann im Hintergrund laufen, auch wenn die App minimiert ist
		\item Content Provider - stellt Daten für die eigene und evtl f"ur andere Apps zur Verf"ugung
		\item Broadcast Receiver - um Systemweite Benachrichtigungen zu empfangen (z.B dass ein Download beendet wurde)
	\end{itemize}
	In der Manifest-Datei werden Eigenschaften der App definiert. Darunter z"ahlen beispielsweisse:
	\begin{itemize}\itemsep0pt
		\item Name der App
		\item Ziel SDK-Versionen
		\item Versionsnummer
		\item optional eine UserId ( siehe \ref*{sec:BasisRechteSystem} )
		\item Permissions ( siehe \ref*{sec:SandBoxingNPermissions} )
	\end{itemize}
	Desweiteren muss jede App signiert werden. Das hierfür benötigte Zertifikat kann sich jeder selbst generieren und muss nicht durch einen Certification Authority ( CA ) beglaubigt werden. Dabei wird angeraten, dass ein Entwickler für all seine Apps dasselbe Zertifikat nutzt.
	
	%03.Sicherheitsarchitektur iOS
	\newpage
	\section{Sicherheitsarchitektur iOS}
	Sicherheitsarchitektur\ldots
	\subsection{Historisches}
	Historisches\ldots
	
	\newpage 
	\section{Sicherheitsaspekte der Android-Architektur}

	Bereits durch die Architektur des Betriebssystems, insbesondere durch die restriktive Rechtevergabe und das Sandboxing, wird versucht ein möglichst sicheres System bereitzustellen.

	\subsection{Basis Rechtesystem}\label{sec:BasisRechteSystem}
	Von Linux wurde auch das Basis-Rechtesystem übernommen. Hierbei bekommt jede App eine eindeutige User-ID (UID) zugewiesen, welche im Normalfall zur Installationszeit zugeteilt wird. Jeder Nutzer, und somit auch jede App, arbeitet grundsätzlich erst einmal nur innerhalb der ihm zugewiesenen virtuellen Maschine und dem damit verbundenen Dateisystem.\\\\
	Da es dennoch in vielen Fällen nötig ist Daten zwischen verschiedenen Apps auszutauschen, gibt es mehrer Möglichkeiten dies zu tun. Die üblichen Wege wären Intends oder SharedPreferences. Zusätzlich gibt es noch die Möglichkeit mehreren Apps dieselbe UID zuweisen zu lassen. Dies ist allerdings nur möglich wenn die entsprechenden Applikationen mit dem selben Zertifikat signiert wurden und in deren Manifest Datei eine gemeinsame UID festgelegt wurde.
	Durch dieses Rechtesystem wird versucht sicherzustellen, dass kein Nutzerprogramm als \textit{root} ausgeführt wird.
	
	\subsection{Sandboxing und Permissions} \label{sec:SandBoxingNPermissions}
	
	%04.Geheime Dienste
	\newpage
	\section{Geheime Dienste}
	Apple kann SMS, Fotos, Videos, Kontakte, Musik, Aufnahmen und Anruferhistorie
	aus passcode geschätzten Geräten auslesen. Möglich machen dies nicht
	dokumentierte Dienste, welche auf jedem Gerät mit iOS installiert sind. In
	diesem Kapitel will ich auf diese Dienste eingehen und deren genauen
	Einsatzzweck erläutern.
	\subsection{lockdownd - remote access}
	Der Dienst \textsl{lockdownd} ermölicht den Zugriff auf ein iOS Gerät
	per TCP oder USB-Anschluss auf Port 62078.
	
	\newpage

\end{document}
