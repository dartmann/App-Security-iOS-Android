%Papierformat
\documentclass[12pt,a4paper]{scrartcl}
%Deutsche Silbentrennung
\usepackage[ngerman]{babel}
%Listen einr�cken
\usepackage{enumitem}
%Deutsche Umlaute
\usepackage[ansinew]{inputenc}
%Trennung von deutschen Umlauten
\usepackage[T1]{fontenc}
%F�r Zitate
\usepackage[numbers,square]{natbib}
%Grafikpaket laden
\usepackage{graphicx}
%Grafik Floating einschr�nken
\usepackage{placeins}
%Referenzierung mit Name
\usepackage{titleref}
%Unterschritszeilen
\usepackage{tabularx}
%Verlinkung
\usepackage{hyperref}
%Urls sollen erkannt werden
\usepackage{url}
%Erm�glicht es von Schrift umflossene Bilder und Tabellen einzuf�gen
\usepackage{wrapfig}

%\makeatletter
%\newcommand\footnoteref[1]{\protected@xdef\@thefnmark{\ref{#1}}\@footnotemark}
%\makeatother​

%Dokumentbeginn
\begin{document}

	%Festlegung des Zitatstyles - Harvardmethode: Abkuerzung Autor + Jahr
	\bibliographystyle{alphadin}

\begin{titlepage}
	%Eine mbox wird verwendet um Text zusammenzuhalten
	%vspace erzeugte die in Klammern angegebenen Zeilenabstände
	%baselineskip setzt zeilenabstand
   	\mbox{}\vspace{5\baselineskip}\\
   	%Schriftart und Größe als Attribut
   	\rmfamily\huge
   	%Mittige Textausrichtung (\centerline für eine Zeile)
   	\centering
   	%Das Argument erscheint in Kapitälchen (small capitals).
	\textsc{Security in iOS}
	%Umbruch bezogen auf die Höhe des Kleinbuchstaben x in diesem Element * Faktor
	\\[3ex]
   	Seminararbeit
   	\rmfamily\Large
   	\vspace{1\baselineskip}\\
   	%Externes einbinden einer Textdatei
   	\input{version.txt}\mbox{}
	\vspace{3\baselineskip}\\
	Hochschule f"ur angewandte Wissenschaften W"urzburg-Schweinfurt
   	\vspace{5\baselineskip}\\
   	\rmfamily\Large
   	David Artmann
   	\vspace{1\baselineskip}\\
   	%Heutiges Datum
   	\today
\end{titlepage}

	%Inhaltsverzeichnis
	\tableofcontents
	\newpage
	%Falls nötig...
	%Literaturliste im Inhaltsverzeichnis anzeigen
	%\addcontentsline{toc}{section}{Literatur}
	
	%Abbildungsverzeichnis
	\listoffigures
	
	%01.Vorwort
	\newpage
	\section{Vorwort}
	Die aus Cupertino in Kalifornien stammende US-Amerikanische Apple Corporation
	ist eine der größten Firmen der Welt und hatte einen Umsatz von 182 Mrd.
	USD im Geschäftsjahr 2014. Sie ist ebenfalls der Erfinder des mobilen
	Betriebssystems iOS, welches auf den firmeneigenen Geräten iPad, iPad mini,
	iPhone, iPod touch und dem Apple TV ab der zweiten Generation zum Einsatz kommt.
	Der Kern des Betriebssystems basiert auf dem freien UNIX Betriebssystem Darwin,
	das auch als Vorlage für das Betriebssystem OS X genutzt wurde.\\
	Im Februar 2015 hatte iOS in den USA einen Marktanteil von 38,8\% und 17,4\%
	in Deutschland.
	\footnote{http://www.kantarworldpanel.com/global/smartphone-os-market-share/}
	Es existieren mehr als einhundert Millionen iPhones auf der ganzen Welt.\\
	Das Smartphone ob iOS, Android oder Windows Phone als Betriebsystem, ist in
	unserer heutigen Welt nicht mehr wegzudenken und mehrere Studien haben offen
	gelegt, dass das Smartphone den Computer bzw. Laptop - zumindest bei der
	Generation unter 18 Jahren - schlägt.
	\footnote{http://www.bitkom.org/files/documents/BITKOM\_PK\_Kinder\_und\_Jugend\_3\_0.pdf}
	\footnote{http://www.mpfs.de/fileadmin/JIM-pdf13/JIMStudie2013.pdf}\\
	Welche Auswirkung hätte es, wenn Sicherheitslücken auf diesen zu finden wären,
	oder schlimmer noch, diese völlig ohne das Wissen des Nutzers ausgenutzt werden
	könnten? Mit dieser Arbeit möchte ich besonders auf Sicherheitstechnische
	Problematiken dieses Betriebssystems eingehen und zeigen, dass ein Proprietäres
	Betriebssystem viele Vorteile hat, aber ebenso auch Nachteile besitzt die es nicht zu verachten gilt. 
	Ebenso werde ich versuchen mit praktischen Beispielen zu zeigen, dass iOS -
	obwohl proprietär - dennoch angreifbar ist und Lücken aufweist.
	
	%02.Apple's law enforcement process guidelines
	\newpage
	\input{ios/pages/02_apple_law_enforcement.tex}
	
	%03.Sicherheitsarchitektur iOS
	\newpage
	\section{Sicherheitsarchitektur iOS}
	Sicherheitsarchitektur\ldots
	\subsection{Historisches}
	Historisches\ldots
	
	%04.Geheime Dienste
	\newpage
	\section{Geheime Dienste}
	Apple kann SMS, Fotos, Videos, Kontakte, Musik, Aufnahmen und Anruferhistorie
	aus passcode geschätzten Geräten auslesen. Möglich machen dies nicht
	dokumentierte Dienste, welche auf jedem Gerät mit iOS installiert sind. In
	diesem Kapitel will ich auf diese Dienste eingehen und deren genauen
	Einsatzzweck erläutern.
	\subsection{lockdownd - remote access}
	Der Dienst \textsl{lockdownd} ermölicht den Zugriff auf ein iOS Gerät
	per TCP oder USB-Anschluss auf Port 62078.
	
	\newpage

\end{document}
