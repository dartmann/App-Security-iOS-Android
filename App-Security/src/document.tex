%Papierformat, mit Koma-Script Dokumentenklasse (Europäisches Design)
\documentclass[12pt,a4paper]{scrartcl}
%Deutsche Silbentrennung
\usepackage[ngerman]{babel}
%Listen einrücken
\usepackage{enumitem}
%Deutsche Umlaute
\usepackage[utf8]{inputenc}
%Trennung von deutschen Umlauten
\usepackage[T1]{fontenc}
%Bibtex für Zitate,
% see: https://en.wikibooks.org/wiki/LaTeX/Bibliography_Management#Customization
\usepackage[]{natbib}
%Grafikpaket laden
\usepackage{graphicx}
%Grafik Floating einschränken
\usepackage{placeins}
%Referenzierung mit Name
\usepackage{titleref}
%Unterschritszeilen
\usepackage{tabularx}
%Verlinkung
% see:https://en.wikibooks.org/wiki/LaTeX/Hyperlinks
\usepackage[hidelinks]{hyperref}
%Urls sollen erkannt werden
\usepackage{url}
%Ermöglicht es von Schrift umflossene Bilder und Tabellen einzufügen
\usepackage{wrapfig}
%Sourcecode anzeigen lassen
\usepackage{listings}
%Farbenpaket laden
\usepackage{xcolor}
%Einrichten des Linking
%\hypersetup{
%    colorlinks,
%    linkcolor={red!50!black},
%    citecolor={blue!50!black},
%    urlcolor={blue!80!black}
%}
%Umbennen der Bibliography Angabe in Literatur
\renewcommand{\bibname}{Literatur}
%Festlegung des Zitierstiles
\bibliographystyle{plain}
%Inhaltsverzeichnis Tiefe erweitern
\setcounter{tocdepth}{5}
%Nummerierung Tiefe erweitern
\setcounter{secnumdepth}{5}

%\makeatletter
%\newcommand\footnoteref[1]{\protected@xdef\@thefnmark{\ref{#1}}\@footnotemark}
%\makeatother​

%Dokumentbeginn
\begin{document}
	
	\input{general/titlepage.tex}

	%Inhaltsverzeichnis
	\tableofcontents
	\newpage
	
	%Abbildungsverzeichnis
	\listoffigures
	
	%Vorwort
	\newpage
	\section{Vorwort}
	Das Smartphone ist in unserer heutigen Welt nicht mehr wegzudenken. Es dient
	als Alltagshelfer mit vielen Funktionen. Es kann als Notizbuch, oder
	Terminkalender genutzt werden. Es ist eines der Hauptkommunikationsmittel, im
	gesprochenen, als auch geschriebenen Wort. Ebenso wird es auch zum
	Zeitvertreib, oder für die Navigation genutzt. Dabei begleitet es uns jeden
	Tag fast überall.
	Mehrere Studien haben offen gelegt, dass das Smartphone den Computer bzw. Laptop - zumindest bei der
	Generation unter 18 Jahren - schlägt.
	\footnote{http://www.bitkom.org/files/documents/BITKOM\_PK\_Kinder\_und\_Jugend\_3\_0.pdf}
	\footnote{http://www.mpfs.de/fileadmin/JIM-pdf13/JIMStudie2013.pdf}
	Das Smartphone hat in den letzten Jahren sich in den Alltag der meisten
	Menschen eingefügt (Statistik?!?!).  Es dient als Alltagshelfer, Notizbuch, 
	Terminkalender, Kommunikationsmittel und Zeitvertreib, und ist dabei fast überall dabei.\\
	Viele vergessen dabei, dass Smartphones mittlerweile die Leistung eines kleinen
	Computers haben und somit auch die selben Gefahren wie am PC Zuhause vorhanden sind.
	Kaum einer hat auf seinem PC kein Anti-Viren System installiert. Aber wer hat
	eines auf seinem Smartphone oder Tablet? Dabei hat man gerade auf diesen Geräten zum Teil 
	hoch sensible Daten gespeichert.\\
	Daher wollen wir im Folgenden auf sicherheitstechnische Aspekte des Android und
	iOS Betriebssystems eingehen und aufzeigen welche Hilfsmittel beide für
	Entwickler und Nutzer bereitstellen.

	
	\newpage
	%Kapitel: Betriebssystem
	\section{Betriebssystem}
	%Das Android Betriebssystem
	
\section{Das Android Betriebssystem}
	Das Unternehmen Android wurde 2003 von Andy Rubin gegründet und wurde 2005 von Google aufgekauft. Seitdem kümmert sich Google und das Android Open Source Project (AOSP) um die Weiterentwicklung des Systems. Zuletzt wurden die neuen Versionen jeweils von Google intern entwickelt und zum Release der Version der AOSP Community als Open-Source bereitgestellt.\\
	Aktuell ist Android, mit 55.6\% Marktanteil \cite{MobileOsStat}, das vorherrschende Betriebssystem für mobile Endger"ate in den USA.
	\\\\
	Basis für das Betriebssystem ist ein modifizierter Linux-Kernel und eine Java Virtual Machine (JVM). Bis einschliesslich Version 4.4 wurde hierfür die Dalivk Runtime und für alle neueren Versionen die Android Runtime (ART) verwendet. Jede App l"auft in einer eigenen Instanz der entsprechenden Runtime und damit in einer Sandbox.\newline
	Oberhalb der JVM sind die meisten Komponenten in Java implementiert.
	
	\begin{figure}[h]
		\centering
		\includegraphics[width=0.7\linewidth]{android_pages/graphics/architektur_android_.png}
		\caption{Die Android Architektur \protect\cite{Elenkov2014} }
		\label{fig:architektur_android}
	\end{figure}
	%Das iOS Betriebssystem
	\section{Das iOS Betriebsystem}
	Die aus Cupertino in Kalifornien stammende US-Amerikanische Apple Corporation
	ist eine der größten Firmen der Welt und hatte einen Umsatz von 182 Milliarden
	US-Dollar im Geschäftsjahr 2014. Sie ist der Erfinder des mobilen
	Betriebssystems iOS, welches auf den firmeneigenen Geräten iPad, iPad mini,
	iPhone, iPod touch und dem Apple TV ab der zweiten Generation zum Einsatz
	kommt. Als Steve Jobs 1985 das Unternehmen verließ, gründete er
	kurze Zeit darauf die Firma NeXT, mit welcher er unter anderem das
	Betriebssystem NeXTStep entwickelte, welches auf dem UNIX ähnlichem
	Betriebssystem BSD\cite[S.12]{Tanenbaum2009} und dem Mach-2.5-Kernel
	\cite{MachProject2015} basiert. NeXT wurde 1996 von Apple aufgekauft und Jobs
	kehrte als CEO zu Apple zurück. Damit begann die Karriere des mobilen
	Betriebssystems iOS. Zuerst wurde NeXTStep als Portierung in Form von Mac OS X
	(später OS X) weiter entwickelt. Mac OS X ist wiederrum der Vorleger für das
	iPhone OS (später iOS), welches am 09. Januar 2007 mit dem damals neu
	erschienenen iPhone erstmals vorgestellt wurde. Im März 2015 hatte iOS in den USA einen
	Marktanteil von 36,5\% und 18,3\% an den insgesamt genutzten mobilen
	Betriebssystemen in Deutschland\cite{MobileOsStat}.
	
	\begin{figure}[h]
		\centering
		\includegraphics[width=0.5\linewidth]{ios/media/marketshare-cmp-201503.jpg}
		\caption{Marktanteil der mobilen Betriebssysteme
		\cite{MobileOsStat}}
		\label{fig:marcetshare}
	\end{figure}
	
	%Apple's law enforcement process guidelines
	\newpage
	\section{Apple's Law Enforcement Process Guidelines}
	Die Firma Apple schreibt auf ihrer
	Webseite\footnote{https://www.apple.com/privacy/government-information-requests/}:
	\begin{quote}
	Our commitment to customer privacy doesn't stop because of a government
	information request.
	\end{quote}
	Weiterhin wird beteuert:
	%TODO: edit this, because the content has changed!!!
	\begin{quote}
	In addition, Apple has never worked with any government agency from any country to create a back 
	door in any of our products or services.
	\end{quote}
	Apple beteuert hier, seine Verpflichtung zur Einhaltung der Privatssphäre des
	Kunden auch nicht einzustellen, wenn die Regierung um Auskunft genau dieser
	Daten bittet.\\
	Zusätzlich wird versichert, dass Apple niemals mit Regierungsbehörden
	jedweder Länder gearbeitet hat, um Trojaner oder andere Hintertüren in eines
	ihrer Produkte oder Dienstleistungen einzubauen.\\
	%TODO: is this really necessary??
	Eine Einhaltung dieser Richtlinien ist von Kundenseite natürlich gewünscht.
	Ob Apple hier wirklich Wort hält, werde ich auf den kommenden Seiten genauer
	beleuchten.

	
	%Grundlegender Aufbau einer Android App
	\newpage
		\section{Grundlegender Aufbau einer Android App}
	\begin{quote}
	Android apps are written in the Java programming language. The Android SDK tools compile your code - along with any data and resource files - into an APK: an \textit{Android package}, which is an archive file with an .apk suffix. One APK file contains all the contents of an Android app and is the file that Android-powered devices use to install the app.
	\end{quote}
	
	Eine App besteht im Kern aus zwei Teilen. Den eigentlichen Programmkomponenten und einer Manifest Datei.
	\\
	Als Programmkomponenten k"onnen unter anderem vorkommen:
	\begin{itemize}\itemsep0pt
		\item Activities - stellen die Benutzeroberfl"ache dar
		\item Services - kann im Hintergrund laufen, auch wenn die App minimiert ist
		\item Content Provider - stellt Daten für die eigene und evtl f"ur andere Apps zur Verf"ugung
		\item Broadcast Receiver - um Systemweite Benachrichtigungen zu empfangen (z.B dass ein Download beendet wurde)
	\end{itemize}
	In der Manifest-Datei werden Eigenschaften der App definiert. Darunter z"ahlen beispielsweisse:
	\begin{itemize}\itemsep0pt
		\item Name der App
		\item Ziel SDK-Versionen
		\item Versionsnummer
		\item optional eine UserId ( siehe \ref*{sec:BasisRechteSystem} )
		\item Permissions ( siehe \ref*{sec:SandBoxingNPermissions} )
	\end{itemize}
	Desweiteren muss jede App signiert werden. Das hierfür benötigte Zertifikat kann sich jeder selbst generieren und muss nicht durch einen Certification Authority ( CA ) beglaubigt werden. Dabei wird angeraten, dass ein Entwickler für all seine Apps dasselbe Zertifikat nutzt.
	
	%Sicherheitsarchitektur iOS
	\newpage
	\section{Komponenten der Systemsicherheit unter
iOS}\label{sec:components-syssec} 
	Eine Kette von aneinander gereihten und von einander abhängigen Prozessen trägt
	maßgeblich zur Systemsicherheit von iOS bei. Dies berücksichtigt vor allem den
	Startvorgang, die Software Updates - auch von Drittanbietern - und den Secure
	Enclave (siehe: Kapitel \ref{sec:secure_enclave}). Dies stellt sicher, dass
	alle Kernkomponenten, ob Hard- oder Software, möglichst gefeit vor Angriffen sind,
	ohne dabei die Nutzerfreundlichkeit zu beeinflussen. Wenn dabei einer dieser
	Schritte fehlschlägt, unterbricht der Startvorgang und das Gerät wird in den
	Recovery Modus versetzt. Wenn der Boot-ROM nicht geladen werden kann, wird der
	DFU (Device Firmware Upgrade) Modus betreten.\\
	Nachfolgend werden die Komponenten welche maßgeblich zur Systemsicherheit und
	an der Wahrung der Integrität dieser beteiligt sind	detailliert vorgestellt und beschrieben.
	
	\begin{figure}[h]
		\centering
		\includegraphics[width=0.4\linewidth]{ios/media/security-model.jpg}
		\caption{Sicherheitsmodel von iOS 
		\cite[S.4]{iOSSecurityApr2015}}
		\label{fig:security-model}
	\end{figure}

	%TODO: look at learning ios security, there this whole process is explained
	% good
	\subsection{Secure boot chain}\label{sec:secure-boot-chain}
		Dieses Verfahren stellt eine Manipulation der Low-Level Software sicher. Nur
		iOS Geräte, die eine erfolgreiche Validierung dieser Vertrauenskette bestanden
		haben, starten ordnungsgemäß. Dabei wird nach dem Start eines iOS Gerätes
		zuerst Code aus einem nur lesbaren Speicherbereich ausgeführt. Dieser
		\textit{hardware of trust} genannte und unveränderbare Code ist bei der 
		Manufaktur der Chips eingebettet worden und somit implizit vertraulich. Das
		Boot ROM enthält zusätzlich den öffentlichen Schlüssel der Wurzel
		Zertifizierungsstelle von Apple, welcher eine Signierung des
		Low-Level-Bootloaders durch Apple sicher stellt, bevor er ausgeführt wird. 
		Dies ist der erste Schritt in der "`chain of trust"', in welcher jeder
		Teilnehmer sicher stellt, dass der darauf folgende von Apple signiert ist. 		
		Nach erfolgreicher Abarbeitung aller Aufgaben des LLB, überprüft und startet
		dieser iBoot, den nächsten Bootloader, welcher wiederrum das selbe Prozedere
		mit dem iOS Kernel startet. Bei Geräten mit Mobilfunk Zugang führt das
		Basisband Untersystem seinen eigenen ähnlichen Prozess mit signierter 
		Software und verifizierten Schlüsseln vom Basisband Prozessor durch.
		
	\subsection{Authorisierung von System Software}\label{sec:code-signing}
		Dieser Prozess soll einen Downgrade auf eine ältere Version von iOS
		verhindern. Der im folgenden Kapitel besprochene Secure Enclave Co-Prozessor
		nutzt diese Technik für Integritätsprüfung seiner Software ebenfalls. Im Falle
		eines iOS Updates wird entweder von iTunes oder vom Gerät selbst einer der
		Apple Installations Authorisations Server kontaktiert. Diesem wird eine Liste
		von verschlüsselten Messungen pro Komponente, welche am Update
		beteiligt ist, gesandt. Zusätzlich wird ein	zufälliger Anti-Replay Wert und
		die eindeutige ID (ECID) des Gerätes verschickt. Diese Auflistung von gegen
		Versionen, für die ein Update genehmigt ist geprüft. Bei Übereinstimmung wird
		die ECID zu den Messungen hinzugefügt und das Ergebnis signiert, was einer
		Personalisierung der Daten gleicht.	Anschließend werden alle signierten Daten
		zum Gerät geschickt. Die sichere Startkette von Prozessen verifiziert die
		Signatur der empfangenen Daten auf den Absender Apple und ob die Prüfsumme der
		Signatur dem entspricht, was das lokale Ergebnis von verschlüsselten Messungen
		und ECID ergibt.\\
		Mit diesen Schritten wird eine Authorisierung nur für bestimmte Geräte sicher
		gestellt. Außerdem verhindert der Anti-Replay Wert ein mitschneiden der Server
		Antwort und das Verwenden der Daten bei anderen Geräten oder ein manipulieren
		der Daten.
	\subsection{Secure Enclave}\label{sec:secure_enclave}
		%TODO: get more details for ensuring the facts in this chapter
		Dieser Co-Prozessor kommt in Geräten mit A7 oder jüngeren A-Serien Prozessoren
		vor. Er verwendet seinen eigenen sicheren Startvorgang, ist separiert vom
		Applikations Prozessor und verwendet verschlüsselten Speicher, sowie einen
		Hardware Zufallszahlen Generator. Die Technologie basiert auf ARM's
		TrustZone\footnote{http://www.arm.com/products/processors/technologies/trustzone/index.php}
		und wurde von Apple für die eigenen Ansprüche angepasst. Der Kernel dieser
		Einheit basiert auf der L4
		Mikrokernel-Familie\footnote{http://os.inf.tu-dresden.de/L4/} mit leichten
		Modifikationen. Die auf Interrupts basierende Kommunikation zwischen dem Applikationsprozessor und dem Secure Enclave (SE) läuft über
		einem nur den beiden zur Verfügung stehenden Speicherbereich. Der SE ist
		verantwortlich für das Schlüssel Management der Datenverschlüsselung und
		stellt die Integrität dieser sicher, auch wenn der Kernel des iOS Systems
		kompromitiert ist. Beim Herstellungsprozess erhält jeder SE eine einzigartige
		ID (UID), auf welche nur er zugreifen kann und welche auch Apple selbst nicht
		bekannt ist. Mit dieser UID wird beim Systemstart der Speicherbereich des SE
		zusammen mit einem einmaligen Schlüssel verschlüsselt. Zusätzlich werden
		jegliche vom SE in den Speicher geschriebene Daten mit der UID und einem
		Anti-Replay Zufallswert verschlüsselt. Eine der Hauptaufgaben des Secure
		Enclave ist die verarbeitung der Fingerabdruck-Daten des Touch
		ID (siehe: Kapitel \ref{sec:touch_id}). Alle Kommunikation zwischen Touch ID
		und SE wird über einen seriellen Bus abgearbeitet. Der Applikationsprozessor leitet die Fingerabdrucksdaten an den
		SE weiter kann diese aber aufgrund einer Verschlüsselung der Daten mit einem
		Sitzungsschlüssel nicht lesen. Dieser Session Key wurde durch den für Touch ID
		und SE bereit gestellten öffentlichen Schlüssel erzeugt. Der Austausch des
		Sitzungsschlüssels wird durch AES Key Wrapping realisiert. Dabei erzeugen
		beide Seiten einen zufälligen Schlüssel, welche den Sitzungsschlüssel bilden.
		Zum verschlüsselten Transport wird AES-CCM genutzt.
	\subsection{Touch ID}\label{sec:touch_id}
		Touch ID bezeichnet den Fingerabdrucksensor der in allen iPhone 5s und neuer
		, sowie iPad Air 2 und iPad mini 3 verbaut ist. Es können bis zu 5
		Fingerabdrücke gespeichert werden. Eine der größten Vorteile dabei ist das
		sofortige Sperren des Gerätes beim drücken des Sleep/Wake-Buttons. Vor der
		Einführung von Touch ID haben viele Nutzer eine möglichst lange Zeit
		eingestellt, bis das Eingeben des Passcodes nötig wurde, nachdem das Gerät
		gesperrt wurde. Dies entfällt bei aktiviertem Touch ID nun, da man nur noch
		mit seinem Finger Entsperren muss. Touch ID kann zusätzlich zum
		Entsperren des Gerätes auch mit dem Zahlungsdienst Apple Pay und für Einkäufe
		in iTunes, dem App Store und im iBook Store genutzt werden. Für Entwickler
		steht eine API bereit mit der aber rudimentärste Prüfungen auf erfolgreiche
		Verifikation des Abdrucks erfolgen können. Ein direkter Zugriff auf Touch ID
		oder die Daten des Fingerabdrucks wird von Apple unterbunden. Touch ID wird
		aktiviert, wenn der kapazitive Stahlring um den Sensor einen Fingerdruck
		wahrnimmt. Dann wird dieser gescannt und an den Secure Enclave geschickt. Der
		Abdruck wird kurzzeitig im veschlüsselten Speicher des SE gespeichert um die
		Daten zu Vektorisieren, danach wird dieser verworfen. Die Schlüssel, welche
		von Touch ID zum Entschlüsseln des Gerätes benötigt werden, sind nach 48
		Stunden ungültig, bzw. wenn das iOS Gerät neu gestartet wurde, oder der
		Fingerabdruck fünf mal falsch registriert wurde.\\
		Dass diese Technik nicht als Sicher angesehen werden darf, haben bereits
		Miglieder des Chaos Computer Club gezeigt
		\footnote{http://www.ccc.de/de/updates/2013/ccc-breaks-apple-touchid}, dabei 
		wurde mit simpelsten Haushaltsmitteln ein Fingerabdruck gefälscht, den das
		Smartphone irrtümlicher weise akzeptiert hat.

	\subsection{iOS}
	Das Kapitel Verschlüsselung unter iOS stellt zu Beginn eingesetzte Hardware vor
	und anschließend rückt der Fokus auf die logische Ebene, mit einem zusätzlichen
	Blick auf den Dateizugriff.
	\subsubsection{Kryptographische Hardware}\label{sec:crypto-engine}
		Für eine native Verschlüsselung sorgt in jedem iOS Gerät eine 256-Bit
		basierte AES Hardware-Engine, welche zwischem dem Flash	Speicher und dem
		Systemspeicher liegt (vgl. Abb. \ref{fig:security-model}). Diese
		Positionierung sorgt für hohe Performance und niedrige Latenzen beim
		Ver- und Entschlüsseln von Dateien. Die geräteeigene Unique ID wird beim
		Herstellungsprozess in den Applikationsprozessor und den Secure Enclave
		eingebrannt und die Group ID wird compiliert, somit ist es keiner Soft- oder
		Firmware möglich diese direkt zu lesen (vgl. Kapitel
		\ref{sec:secure-boot-chain}). Dies verhindert eine Manipulation oder ein
		Umgehen dieser Schlüssel, sowie einen Zugriff außerhalb der Krypto-Engine.
		Die UID erlaubt außerdem eine gerätespezifische Bindung von Daten. Apple
		beteuert unter anderem
		\begin{quote}
			The UIDs are unique to each device and are not recorded by Apple or any of its
			suppliers \cite[S.9]{iOSSecurityApr2015}.
		\end{quote}
		Die GID hingegen ist allen Geräten einer
		Prozessorklasse bekannt, z.B.
		all jenen die einen Apple A7 nutzen. Dies liegt an der weniger
		Sicherheitskritischen Nutzung dieser Schlüssel für Vorgänge,
		wie beispielsweise die Übertragung von Systemsoftware bei Updates. Jegliche
		weitere bei Kryptografischen Operationen benötigte Schlüssel werden von einem
		Random Number Generator (RNG) mit einem auf	CTR\_DRBG \cite{NISTDRBG2012}
		basierenden Algorithmus erzeugt.
	\subsubsection{Schutz auf Dateiebene}\label{sec:filesecurity}
		Zusätzlich zur Hardwareverschlüsselung nutzt iOS ein Feature namens
		\textsl{Data Protection} um Daten im Flash Speicher zu schützen. Dieser
		Mechanismus wird bei allen System Apps und ab iOS 7 auch bei Drittanbieter Apps
		automatisch angewendet. Die Funktionsweise ist in Form einer
		Schlüsselhierarchie im Verbund mit Hardwareverschlüsselung realisiert. Jede
		Datei, die in den Speicher geschrieben wird, erhält einen 256-Bit, explizit
		ihr zugewiesenen, \textsl{Per-File} Schlüssel. Dieser wird von der
		Verschlüsselungsengine (Kapitel \ref{sec:crypto-engine}) mit Hilfe von
		AES CBC (vgl. Kapitel \ref{sec:encrypt-volume})
		verschlüsselt. Der Per-File Schlüssel wird mit einem von vier
		Klassenschlüsseln ummantelt und in den Meta-Daten gespeichert. Die vier
		möglichen Schlüssel sind:
		\begin{itemize}
		  \item \textsl{Complete Protection}: Klassenschlüssel wird mit einem
		  Schlüssel, abgeleitet aus passcode und UID, verschlüsselt. 10 Sekunden (wenn
		  die "`Passwort benötigt"' Funktion ohne Verzögerung eingestellt wurde) nach
		  Sperren des Gerätes wird dieser Schlüssel verworfen.
		  \item \textsl{Protected Unless Open}: Während das Gerät gesperrt ist, können
		  bestimmte Dateien, wie beispielsweise ein E-Mail Anhang der heruntergeladen
		  wird, dennoch geschrieben werden. Um dies zu ermöglichen wird
		  Elliptische-Kurven-Kryptographie eingesetzt (ECDH\footnote{ECDH: Elliptic
		  curve Diffie–Hellman - ein auf elliptischen Kurven basierendes
		  Schlüsselaustauschverfahren} mit Curve25519\footnote{Curve25519 - eine
		  elliptische Kurve, welche auch für Schlüsselaustauschprotokolle genutzt
		  wird}).
		  \item \textsl{Protected Until First User Authentication}: Der Unterscheid zu
		  \textsl{Complete Protection} besteht darin, dass der genutzte Schlüssel
		  beim Sperren des Gerätes nicht verworfen wird. Diese Klasse wird per
		  Standard für Drittanbieter Apps verwendet, wenn nicht explizit angepasst.
		  \item \textsl{No Protection}: Hier wird der Klassenschlüssel nur mit der UID
		  verschlüsselt und im \textsl{Effaceable
		  Storage}\footnote{Effaceable Storage - dedizierter Bereich im NAND-Speicher,
		  welcher direkt adressiert und sicher gelöscht werden kann} gespeichert.
		\end{itemize}
		Jegliche Meta-Daten sind mit einem zufälligen Schlüssel verschlüsselt, welcher
		beim installieren von iOS oder beim Löschen eines Gerätes durch den Benutzer
		erstellt wird. Beim Entschlüsseln einer Datei, werden zuerst die Meta-Daten
		mit dem \textsl{File System Key} entschlüsselt. Dieser ist im Effaceable Storage
		gespeichert.
		Sobald dieser Speicher und somit der darin enthaltene Dateisystemschlüssel
		gelöscht wurde, macht dies alle Dateien auf dem Gerät kryptografisch
		unwiederherstellbar. Wenn der Benutzer \textsl{passcode} auf dem System
		einrichtet, aktiviert er Data Protection automatisch.
		\begin{figure}[h]
			\centering
			\includegraphics[width=0.9\linewidth]{ios/media/data-protection.jpg}
			\caption{Data Protection Architektur 
			\cite[S.10]{iOSSecurityApr2015}}
			\label{fig:data-protection}
		\end{figure}

	
	%Sicherheit Android
	\newpage 
	\section{Sicherheitsaspekte der Android-Architektur}

	Bereits durch die Architektur des Betriebssystems, insbesondere durch die restriktive Rechtevergabe und das Sandboxing, wird versucht ein möglichst sicheres System bereitzustellen.

	\subsection{Basis Rechtesystem}\label{sec:BasisRechteSystem}
	Von Linux wurde auch das Basis-Rechtesystem übernommen. Hierbei bekommt jede App eine eindeutige User-ID (UID) zugewiesen, welche im Normalfall zur Installationszeit zugeteilt wird. Jeder Nutzer, und somit auch jede App, arbeitet grundsätzlich erst einmal nur innerhalb der ihm zugewiesenen virtuellen Maschine und dem damit verbundenen Dateisystem.\\\\
	Da es dennoch in vielen Fällen nötig ist Daten zwischen verschiedenen Apps auszutauschen, gibt es mehrer Möglichkeiten dies zu tun. Die üblichen Wege wären Intends oder SharedPreferences. Zusätzlich gibt es noch die Möglichkeit mehreren Apps dieselbe UID zuweisen zu lassen. Dies ist allerdings nur möglich wenn die entsprechenden Applikationen mit dem selben Zertifikat signiert wurden und in deren Manifest Datei eine gemeinsame UID festgelegt wurde.
	Durch dieses Rechtesystem wird versucht sicherzustellen, dass kein Nutzerprogramm als \textit{root} ausgeführt wird.
	
	\subsection{Sandboxing und Permissions} \label{sec:SandBoxingNPermissions}
	
	%Applikationssicherheit unter iOS
	\newpage
	\section{Applikationssicherheit unter iOS}
	Mobile Betriebssysteme bauen zum größten Teil auf Applikationen - weiter
	\textsl{Apps} genannt - auf, welche die produktive Nutzung eines mobilen
	Endgerätes erheblich verbessern können. Dabei ist es allerdings
	essentiell, dass diese Apps korrekt vom Betriebssystem behandelt werden, da
	andernfalls die Systemsicherheit, die Stabilität oder gar die Nutzerdaten
	gefährdet werden können. Wie bereits im Kapitel Systemsicherheit (siehe:
	\ref{sec:components-syssec}) vorgestellt, wird auch hier eine Art
	Schichtensystem angewendet, um eine Signierung und Verifikation, sowie ein
	Sandboxing der Apps sicherzustellen.
	\subsection{Signieren von Applikationen}
		Nach dem Start des iOS Kernels, stellt dieser sicher, welche Nutzerprozesse
		und Apps gestartet werden dürfen. Dazu werden diese auf eine Signierung durch
		ein von Apple ausgestelltes Zerifikat geprüft. Das zwingende Vorhandensein
		dieses Zertifikats stellt eine Adaption der \textsl{chain of trust} (siehe:
		\ref{sec:secure-boot-chain}) auf die Applikationsebene dar. So muss jeder
		private Entwickler als auch jedes Unternehmen seine Identität bei Apple
		verifizieren, bevor ein Entwicklerzertifikat von Apple für diese ausgestellt
		wird. Somit ist sicher gestellt, dass jede App im AppStore auf eine
		Privatperson zurückzuführen ist, was auch ein gesteigertes Vertrauen der
		Nutzer in die Qualität der Apps zur Folge hat.\\
		Allerdings muss an diesem Punkt erwähnt werden, dass Apple Ausnahmen dieser
		Verifikation in Form des \textsl{iOS Developer Enterprise
		Program}\footnote{https://developer.apple.com/programs/ios/enterprise/}
		erlaubt und so Apps auch vorbei am AppStore und auf iOS Geräte installiert
		werden können. Dabei prüft Apple das anfragende Unternehmen auf Eignung durch
		deren D-U-N-S Nummer - einem Zahlensystem zur eindeutigen Identifizierung von
		Firmen. Populär wurde eine jüngste Ausnutzung dieses Privileges, bei welcher
		über eine Webseite bei Einwilligung eine App installiert wird, welche dann
		ein Abonnement verkaufen will\footnote{http://heise.de/-2679222}. Apple
		dachte diese Möglichkeit nur für Firmen, die ihr eigenes Mobile Device
		Management betreiben, an. Hier wurde also entweder der ausstellende Account
		gekapert oder diese Lizenz vom Eigentümer schlichtweg missbraucht.\\
		Ab iOS8 wird es Entwicklern erlaubt in ihren Apps Frameworks zu verwenden. Um
		hier ein Laden von unsigniertem Code zu verhindern, wird beim Start einer App
		die \textsl{Team-ID} geprüft - ein 10 stelliger alphanumerischer String,
		welcher aus dem von Apple ausgestellten Entwicklerzertifikat extrahiert wird.
		Eine App darf nur Code laden, welcher entweder vom System kommt, oder die selbe
		Team-ID besitzt.
	\subsection{}
	
	%Proprietaet und undokumentierte Dienste
	\newpage
	\subsection{Proprietät von iOS}\label{sec:proprietaer-ios}
	Definition Proprietär:
	\begin{quote}
		Proprietäre Software (lateinisch proprie "`eigentümlich"', "`eigen"',
		"`ausschließlich"') wird eine Software bezeichnet, die das Recht und die
		Möglichkeiten der Wieder- und Weiterverwendung durch Dritte stark einschränkt
		\cite{WikiProprietary2015}.
	\end{quote}
	Im Gegensatz zu anderen Herstellern mobiler Betriebssysteme lizenziert Apple
	iOS nicht für andere Hardwarehersteller, sondern produziert alles im
	eigenen Hause. IOS wird somit nur auf Hardware aus dem eigenen kontrollierten
	Herstellungsprozess eingesetzt. Dieser proprietäre Ansatz birgt gewisse
	Gefahren(Kapitel
	\ref{sec:undocumented-services}), die in diesem Kapitel beleuchten werden
	sollen. Außerdem hat sich gezeigt, dass die Nutzergemeinde versucht aus der
	Proprietät zu entkommen bzw. sie zu umgehen (Kapitel \ref{sec:jailbreaking}).
	
	\subsubsection{Undokumentierte und kritische
	Dienste}\label{sec:undocumented-services}
		%TODO: rewrite this header
		% Drei Dienste wurden erst im Juni 2013 offiziell dokumentiert, siehe:
		% https://support.apple.com/en-us/HT203034 und das auch erst nachdem sie
		% zdziarski in seinem paper gerügt hat.
		Apple kann SMS, Fotos, Videos, Kontakte, Musik, Aufnahmen und Anruferhistorie
		aus passcode geschützten Geräten auslesen. Möglich machen dies nicht
		dokumentierte Dienste, welche auf jedem Gerät mit iOS installiert sind. In
		diesem Kapitel will ich auf diese Dienste eingehen und deren genauen
		Einsatzzweck erläutern.
		\paragraph{lockdownd - remote access}
			Der Dienst \textsl{lockdownd} ermöglicht den Zugriff auf ein iOS Gerät
			%TODO: check if lockdownd also could be connected over wifi?!
			per TCP auf Port 62078 über einen USB-Anschluss.
			Dies wird über das eigene Protokoll \textsl{usbmux} gehandhabt. Dieses
			erlaubt TCP Verbindungen zum lokalen System über einen USB Anschluss mit
			%TODO: double check this!
			übergebenen	Portnummern. Der lockdownd Deamon erstellt nach eingehender
			Anfrage eine TCP-Verbindung auf localhost unter dem angegebenen Port.
		\paragraph{com.apple.mobile.pcapd}
			Pcapd stellt einen Überwachungsdienst dar, welcher auf der
			Programmierschnittstelle
			pcap \cite{PCAP2015} aufbaut und
			diese in Form der Bibliothek \textsl{libpcap} implementiert. Mit diesem
			Dienst ist es möglich jeglichen Netzwerkverkehr mitzulesen und zu speichern.
			Snifferprogramme wie Whireshark \cite{WHIRESHARK2015}, greifen auf
			diese Schnittstelle zurück, um Pakete direkt an der Netzwerkschnittstelle
			abzufragen. Dem Benutzer des iOS Gerätes wird es visuell nicht erläutert,
			wenn dieser Dienst aktiviert ist. Zusätzlich ist kein Entwickler-Modus von
			nöten, um diesen zu aktivieren. Seit iOS 8 ist es nicht mehr möglich diesen
			Dienst über WiFi anzusprechen.
		\paragraph{com.apple.mobile.file\_relay}
			Hinter diesem Paket verbirgt sich ein Dienst mit dem auf sensible und
			persönliche Daten zugegriffen werden kann. Darunter auch das Adressbuch, GPS
			Daten, Fotos und der Möglichkeit ein Image der Metadaten des Dateisystems
			abzugreifen. Apple schreibt dazu:
			\begin{quote}
				In iOS 8 and later, this capability requires additional configuration before
				use \cite{AppleDiagnosticCap2015}.
			\end{quote}
			Was eventuell auf eine Verbesserung des Zugriffs auf diesen Dienst hinweisen
			könnte. Beispielsweise nur noch authorisierter vom Benutzer genehmigter
			Zugriff.
		\paragraph{com.apple.mobile.house\_arrest}
			Apple dokumentiert hier eine Verwendung von iTunes für den Transfer von Daten
			mit einem iOS Gerät. Zusätzlich wird dieses Feature von Xcode verwendet,
			wenn Testdaten zu einem Gerät geschickt werden sollen und sich die relevante
			%TODO: full name plus reference of his paper!!!
			App im Entwicklermodus befindet. Laut Zdziarski hat dieser Dienst Zugriff auf
			die Library, den Cache, Cookies und bevorzugte Ordner, welche hoch sensible
			Daten beinhalten - wie den Zwischenspeicher des sozialen Streams von Facebook
			oder Twitter - obwohl die iTunes GUI dies nicht erlaubt.
			
	\subsubsection{Historische Exploits}
		In der Geschichte von iOS hat es bereits diverse Exploits \cite{Exploit2015}
		gegeben. In diesem Kapitel sollen einige der bekanntesten vorgestellt werden.
		\paragraph{libTiff Exploit} 
			Eine auf einem Pufferüberlauf der libtiff Bibliothek basierenden
			Sicherheitslücke. Dabei wurde die genannte Bibliothek gepatcht, um einen
			Jailbreak zu starten \cite{LibTiffExploit2015}. Dieser Exploit
			stammt aus der Anfangszeit von iOS in der alle Prozesse noch mit root
			Rechten liefen.
		\paragraph{Ikee Virus}\label{sec:ikee-exploit}
			Dabei handelt es sich um einen der ersten Würmer unter iOS. Der	Author Ashley Towns
			wollte 2009 damit auf nicht veränderte Standardpasswörter der SSH Zugänge,
			bei iPhones mit Jailbreak hinweisen. Er nutzte mit dem in C geschriebenen
			Programm lediglich das Standardpasswort \textsl{alpine} des bei jailbroken
			meist mit installierten SSH Dienstes, deaktivierte bei erfolgreichem login
			den SSH Dienst und versuchte dann das Skript auf zufällig ausgewählten
			IP-Adressen auszuführen \cite{IkeeExploit2009}. Eine unter dem Namen
			\textsl{Ikee.B} veränderte Variante stahl Inhalte des Gerätes und machte
			dieses zum Zombie und somit Teil eines Botnetzes.
		\paragraph{SpyPhone}
			%TODO: maybe reference to equal android example???
			Diese App wurde vom Author Seriot Nicolas als
			\textsl{proof of concept} erstellt. Dabei nutzte er nur die gegebenen
			legitimen API's zum Erstellen von Apps für den AppStore. Obwohl diese App
			komplett innerhalb ihrer von iOS vorgeschriebenen Sandbox lief, konnte
			Sie dennoch:
			\begin{itemize}\itemsep0pt
				\item{Die Nummer des Gerätes auslesen}
				\item{Vom Adressbuch lesen und ebenfalls darin schreiben}
				\item{E-Mail Kontoinformationen auslesen}
				\item{Den Zwischenspeicher der Tastatur mitlesen}
				\item{Auf das GPS zugreifen}
				\item{WLAN Zugangspunkte auslesen}
			\end{itemize}
			
	\subsubsection{Jailbreaking}\label{sec:jailbreaking}
		%TODO: check writing
		Der aus dem englischsprachligen Raum stammende Begriff bezeichnet unter iOS
		das gezielte Aushebeln der von Apple vorgegebenen Restriktionen. Oft werden
		dabei eigene Paketverwaltungstools wie
		Cydia \cite{Cydia2015} installiert, um auch von Apple nicht
		signierte Apps auf die modifizierten Geräte aufspielen zu können.\\
		Es werden zwei Arten von Jailbreaks unterschieden, der \textsl{tethered
		jailbreak} und der \textsl{untethered jailbreak}. Dabei bedeuted tethered,
		dass eine Anbindung an einen Computer, auf welchem eine Jailbreaking Software
		- beispielsweise \textsl{redsn0w}, oder	\textsl{TaiG} - installiert sein muss, die
		den Prozess unterstützt. Untethered Jailbreaks kommen ohne diese Anbindung aus
		und arbeiten autark. Das Prinzip des Jailbreakings ist dabei immer das selbe:
		es wird mit Hilfe eines Exploits das System kompromittiert. Zu Beginn wurden oft
		Hardwarefehler ausgenutzt, die während des
		DFU-Modus (Kapitel \ref{sec:components-syssec}) installiert wurden.
		Später kamen \textsl{Userland-Exploits} dazu, welche während dem
		laufenden Betrieb von iOS gestartet werden und sich Schwachstellen im
		Betriebssystem zu nutze machen.
		%TODO: extend this section with more details
	
	\subsubsection{Vorteile der Proprietät}
		Obwohl der Nutzer unter iOS dazu gezwungen wird, sich den Vorgaben durch Apple
		zu beugen, hat es auch seine Vorteile, dies zu akzeptieren. So sind bei
		aktivem Jailbreak Sicherheitsmechanismen wie die Signaturprüfung (Kapitel
		\ref{sec:code-signing}), oder die Apple Secure Boot	Chain (siehe: Kapitel
		\ref{sec:secure-boot-chain}) möglicherweise modifiziert beziehungsweise
		deaktiviert worden, was ein enormes Sicherheitsrisiko darstellen kann.
		Außerdem haben uns Exploits wie Ikee (Kapitel \ref{sec:ikee-exploit})
		gezeigt, dass zu viel Macht an falscher Stelle, verbunden mit Unwissenheit,
		meist mehr Schaden anrichtet, als Besserung zu bringen. Das BSI warnt
		zudem in einem Überblickspapier zu iOS offiziell vor dem Einsatz von
		Jailbreaks \cite{BSIJailbreak2013}.

	
	%Rooting von Android Geräten
	\newpage
	\section{Rooting von Android-Geräten}

testalödksjföasdfölaskdjföl
	
	%Sicherheit durch Open-Source
	\newpage
	\section{Android: Sicherheit durch Open-Source}
Das Android Betriebssystem wird in erster Linie von Google und dem Android Open Source Projekt gewartet und weiterentwickelt. Durch dieses Vorgehen erhofft man sich eine stetige Verbesserung des Systems, sowohl aus funktionaler, wie auch sicherheitstechnischer Sicht. \\
Durch ein öffentliches Bereitstellen des Source Codes soll auch die Wahrscheinlichkeit für Sicherheitslücken minimiert werden, da somit mehr Entwickler die Möglichkeit bekommen sich diesen anzuschauen und Sicherheitslücken zu entdecken. Dass dies allerdings nicht garantiert ist, wurde zuletzt durch den Heartbleed-Bug in der SSL Bibliothek OpenSSL bekannt. Obwohl OpenSSL Open-Source ist, und viele Entwickler daran arbeiten, konnte die Schwachstelle über längere Zeit hinweg unentdeckt bleiben. Dennoch wird es, durch das öffentliche Bereitstellen des Programmcodes, deutlich schwerer versteckte Programmierschnittstellen mit einzubauen, wie es beispielsweiße in iOS der Fall ist.\\
Ein großes Plus ist, dass dadurch, dass ein Linux-Kernel die Basis von Android ist, die stetige Entwicklung des Kernels sich auch in die Entwicklung des Android Betriebssystems niederschlägt und umgedreht.\\
Allerdings darf nicht vergessen werden, dass die meisten Smartphone Hersteller Android für ihre Geräte erst noch modifizieren und Closed-Source Apps mit erweiterten Rechten installieren können. Solche Modifikationen sind zumeist nicht Open-Source. Zum Beispiel sind in dem offiziellen Android Release von Google, immer einige Google eigene Apps vorinstalliert, deren Source Code nicht einsehbar ist. So ist mittlerweile bekannt geworden, dass über die App \textit{Google Einstellungen} (engl. \textit{Google Settings}) Google in der Lage ist, per Fernzugriff Applikationen zu löschen und zu installieren. Dies wurde bekannt, nachdem eine App eines Forschers aus dem App Market (heute Play Store) entfernt wurde und die Funktion zum Löschen der Apps genutzt wurde um restliche Installationen zu entfernen.\footnote{http://android-developers.blogspot.de/2010/06/exercising-our-remote-application.html} 
Diese Funktion ist angedacht, um Schadcode von infizierten Geräten zu entfernen, kann aber auch genutzt werden um Apps zu installieren.\footnote{http://www.computerworld.com/article/2506557/security0/google-throws--kill-switch--on-android-phones.html} Aktuell ist nicht bekannt, was über diese App noch steuerbar ist.\\
Bei solchen Funktionen stellt sich immer die Vertrauensfrage. Der eigentliche Sinn und Zweck bringt Komfort und schnelle Abhilfe. Die Frage ist nur, wie gut solche Funktionen abgesichert sind und ob sie für andere Zwecke von dritter Missbraucht werden können. Selbst wenn derartige Applikationen gegen den unberechtigten Zugriff dritter abgesichert ist, bleibt noch offen ob staatliche Institutionen von Firmen wie Google Zugang bekommen oder erzwingen können. Wie groß die Angst vor derartigen Zugriffen ist, kann man an der seit gut zwei Jahren laufenden Überwachungsaffäre, welche durch Edward Snowden angestoßen wurde, erkennen.
	
	\newpage
	\input{android_pages/Update_Problem.tex}
	
	%Härten von iOS
	\newpage
	\section{Härten von iOS}
	Trotz aller vorhandenen Hilfsmittel sollte man nicht auf die Sicherheit von iOS
	vertrauen. Deshalb liste ich abschließend je ein Unterkapitel für Entwickler
	und für Endnutzer, in welchen mögliche Verbesserungen für die
	Entwicklung einer App oder für das Einrichten von iDevices gelistet werden.
	\subsection{Relevantes für den Entwickler}
		Der Entwickler ist verantwortlich für die Sicherheit seiner Anwendung und
		sollte jegliche mögliche Option der zusätzlichen Absicherung seiner
		Applikation in Erwägung ziehen. Die folgenden Unterkapitel stellen eine
		Auflistung der wichtigsten Maßnahmen zum Schutz der eigens entwickelten App
		dar. An dieser Stelle gilt es zu erwähnen, dass es sich hier nicht um nicht
		alle Möglichkeiten handelt und vor allem, dass keine Applikation absolut
		sicher gemacht werden kann. Ein Einsetzen der folgenden Herangehensweisen
		steigert die Sicherheit aber dennoch und kann die Zeit, welche für einen
		erfolgreichen Angriff nötig ist, um ein Vielfaches erhöhen.
		\subsubsection{Passwortstärke}
			Die Verschlüsselung ist eine der wichtigsten Arten seine Daten zu schützen,
			aber auch die kritischste, wenn es um die Implementierung geht. Daher richten
			Hacker ihr Augenmerk zuerst auf die Implementierung und nicht die
			eigentlichen verschlüsselten Daten. Hierbei ist es besonders wichtig, keine
			schwachen Passwörter zu erlauben. Eine strenge Passwortrichtlinie ist
			Pflicht, auch wenn es die Useability negativ beeinflusst. Die gewählten
			Passwörter sollten aus vielen Stellen bestehen (mindestens 12),
			welche wiederrum zu einem Anteil aus Zahlen, Sonderzeichen und Zeichen, in
			Groß- und Kleinschreibweise, bestehen. Zusätzlich sollten bestimmte Muster,
			wie entlang der QUERTZ-Tastatur zu fahren, einfache Wörter, welche meist in
			Dictionaries enthalten sind und strukturierte Daten wie ein Datum, verhindert
			werden.
		\subsubsection{Common Crypto Library}
			IOS bietet mit der \textsl{Common Crypto Library}\footnote{https://goo.gl/5ApuXL}
			(3CC oder auch CCCrypt) eine Möglichkeit auf C-Ebene 
			Verschlüsselungsalgorithmen wie AES, DES oder 3DES einzusetzen. Dabei bietet
			3CC je nach eingesetztem Algorithmus Block- beziehungsweise Stromchiffre an.
			Zusätzlich wird mit dem \textsl{Cipher Block Chaining} (CBC) eine
			Möglichkeit angeboten, um Man-In-The-Middle, sowie Replay-Angriffe zu
			verhindern. Dies ist möglich, da bei CBC jeder Klartext-Block mit dem
			vorherigen verschlüsselten Chiffre XOR-verknüpft wird und anschließend
			ebenfalls chiffriert wird. Somit ist jeder Block von der bisherigen Ketten
			verschlüsselter Daten abhängig.
		\subsubsection{Sicherung des Hauptschlüssels}\label{sec:master-key}		
			Falls ein Master-Key - also ein übergeordneter Schlüssel - zur
			Verschlüsselung eingesetzt wird, sollte dieser zwingend ebenso verschlüsselt
			werden. Das National Institute of Standards and Technology (NIST)
			empfiehlt\cite{NISTPBKDF2010} dazu Passwort basierte 
			Schlüsselableitungsfunktionen (Password-Based Key Derivation Function),
			allgemein unter dem Kürzel \textsl{PBKDF2} bekannt.
			Diese leiten die Eingabe über mehrere, beziehungsweise eine gewünschte Anzahl
			von Iterationen zu einem Schlüssel der gewünschten Komplexität ab. Dieses
			Ergebnis wird dann genutzt, um den Master-Key zu verschlüsseln. Der
			eigentliche Vorteil bei dieser Herangehensweise liegt darin, dass der
			Master-Key nie geändert werden muss. Wenn der Nutzer beispielsweise sein
			Passwort ändert, muss der Master-Key nur mit dem neuen Passwort abgeleitet
			über die PBKDF verschlüsselt werden. Außerdem ist es bei diesem Ansatz
			möglich mehrere Kopien des Hauptschlüssel in verschiedenen Wegen
			abgespeichert werden können. Ein Anwendungsbeispiel stellt die
			Verwendung von Rücksetzungsmechanismen von Passwörtern dar, wobei der
			Benutzer aufgefordert wird die Antwort auf eine bestimmte Frage zu geben,
			welche er zur Erstellung des Passwortes festgelegt hat.
		\subsubsection{Lokations- und Tempusberücksichtigung}
			Sogenannte Geo-Encryption bringt einen weiteren Faktor in die Kette der
			möglichen Verbesserungen des Schutzes für unsere Applikation. Damit wird
			ein gewisser Aufenthaltsort, oder ein Radius um diesen vorgeschrieben. Das
			Gerät muss sich in diesem befinden, um eine Entschlüsselung zu ermöglichen.
			Zusätzlich kann die Zeit immer eine weitere Optimierung sein, indem die
			Entschlüsselung nur in einem gewissen Zeitfenster erlaubt wird.
		\subsubsection{Bipartite Schlüssel}
			Diese Vorgehensweise stellt eine - ähnlich zur Geo-Verschlüsselung - zweite
			Abhängigkeit im Entschlüsselungsprozess dar. Hierbei werden zwei Schlüssel
			beim erstmaligen Start der Applikation erzeugt und mit einander
			XOR-verknüpft. Dieses Ergebnis wird dann zum Verschlüsseln des
			Hauptschlüssels (siehe: \ref{sec:master-key}) verwendet. Einer der beiden
			Schlüssel wird durch die Passphraseeingabe des Nutzers erzeugt wird und der
			andere ein Zufallswert ist, welcher an einen entfernten Server geschickt
			wird. Beim Start der App muss sich der Benutzer sowohl durch seine
			gewählte Passphrase lokal authentifizieren, als auch beim Server anmelden, um
			den dort gespeicherten Zufallswert zu erhalten. Diese beiden Werte werden
			anschließend XOR-verknüpft, um damit den Master-Key zu entschlüsseln, mit
			welchem die Daten entschlüsselt werden. Falls ein Gerät als
			gestohlen vermutet wird, kann hier einfach der Serverseitige Schlüssel verworfen
			werden und jegliche Bemühungen des Angreifers, an die Daten zu kommen, wären
			vergebens.
		\subsubsection{Manipulationsschutz}
			Wenn ein Gerät gestohlen wurde, gibt es verschiedene Möglichkeiten den
			Schaden minimal zu halten und weiteren zu verhindern. Das löschen von lokalen
			Benutzerdaten zählt zu diesen Maßnahmen. Dabei muss im Idealfall nur der
			Master-Key, welcher die relevanten Daten verschlüsselt,	überschrieben,
			beziehungsweise gelöscht werden und die Nutzerdaten sind unwiederherstellbar
			für den Angreifer. Diese Herangehensweise ist unauffälliger, als
			beispielsweise ein komplettes "`Nullen"' der Datenbank und kostet den
			Angreifer frustrierende Zeit.\\
			Im Falle einer Manipulation muss davon ausgegangen werden, dass der Benutzer
			nicht mehr vertrauenswürdig ist. Somit sollte eine eventuell vorhandene
			Anbindung an entfernte Resourcen (z.B. Server, welche Daten für die App
			speichern) unterbrochen werden. Dies kann bereits mit dem Setzen eines Flags
			in einer Konfigurationsdatei, oder dem Deaktivieren der Credentials des
			Nutzers auf Serverseite realisiert werden. Auf diese Weise handeln wir
			präventiv gegen weitere Schäden, ohne dass der Angreifer davon etwas
			erfährt.\\
		\subsubsection{Jailbreakerkennung}
			Es kann erwünscht sein, dass die entwickelte App nicht auf
			\textsl{jailbroken} iOS Geräten installiert werden darf. Da man sich bei
			diesen Geräten nicht mehr auf den nativen von iOS gegebenen Schutz (siehe:
			Kapitel \ref{sec:components-syssec}) verlassen kann (siehe: Kapitel
			\ref{sec:jailbreaking}). Diese Erkennung wird über einen
			Integritätstest der Sandbox mit dem Ausführen des Befehls \textsl{fork}
			getestet\cite[S.328]{Zdziarski2012}. Dieser erlaubt der App einen
			Kinds-Prozess zu starten. Wenn die Sandbox kompromittiert wurde, oder die App
			außerhalb dieser läuft, wird die Operation erfolgreich ausgeführt. Folgender
			Programmcodeauszug zeigt besagten Vorgang:
			\lstdefinestyle{customc}{
				belowcaptionskip=1\baselineskip,
			  	breaklines=true,
			  	frame=L,
			  	xleftmargin=\parindent,
			  	language=C,
			  	showstringspaces=false,
			  	basicstyle=\footnotesize\ttfamily,
			  	keywordstyle=\bfseries\color{green!40!black},
			  	commentstyle=\itshape\color{purple!40!black},
			  	identifierstyle=\color{blue},
			  	stringstyle=\color{orange},
			}
			\lstinputlisting[language=C]{ios/code/jailbreakdetection.c}
	\subsection{Tips für den Benutzer}
		Als Endnutzer hat man ebenfalls eine gewisse Fülle an Möglichkeiten seine
		Daten zu schützen und bedächtig vorzugehen. Dies bezieht sich in diesem Fall
		aber auf die Ebene des User Interfaces und dessen Einstellungen.
		\subsubsection{Lockscreen einschränken}
			In früheren iOS Versionen gab es bereits mehrere Male die Möglichkeit den
			gesperrten Bildschirm des Gerätes zu umgehen
			\footnote{https://www.exploit-db.com/exploits/28978/}\footnote{https://www.youtube.com/watch?v=NTA8k4tyY78}.
			Dies kann verhindert werden indem dafür nötige Dienste - zumindest für den Lockscreen
			- deaktiviert werden. Wenn die Sprachsteuerung \textsl{Siri} nicht benötigt
			wird, sollte diese ebenfalls für den Sperrbildschirm deaktiviert werden.
			Generell gilt hier ein Weniger-ist-Besser-Vorgehen.
		\subsubsection{iOS 9}
			Zum Zeitpunkt des Entstehens dieser Arbeit befindet sich iOS 9 noch im
			Entwicklungsstatus. Dennoch lohnt ein Blick auf ein kommendes Feature dieses
			jüngsten Apple Sprößlings: der Zwei-Wege-Authentifizierung. Diese wird mit
			iOS 9 nativ eingeführt und fordert den Nutzer - sofern aktiviert - dazu auf
			beim Einloggen in seinen Apple ID Account von einem bisher unbekannten
			Browser oder Gerät einen Verifizierungsschlüssel einzugeben, der auf seinen
			restlichen iOS Geräten angezeigt
			wird\footnote{https://www.apple.com/ios/ios9-preview/}.
		%TODO: check other possibilities

	
	%Härten von Android
	\newpage
	\section{Härten von Android}
\subsection{Für Entwickler}

\subsection{Für Endnutzer}
	
	%Schlusswort
	\newpage
	\section{Schlusswort}
	Auf den ersten Blick erscheinen Android und iOS wie absolut differente mobile
	Betriebssysteme. Bei näherem Hinsehen wird aber schnell ersichtlich, dass diese
	in vielen Bereichen den selben Ansätzen folgen. Dabei liegt es in der Natur der
	Sache, dass es immer Sicherheitsmängel geben wird, da der Mensch an sich Fehler macht. 
	Ein stetiger Wandel bestimmt beide Plattformen, welcher auch
	sicherheitstechnische Aspekte betrifft. Dabei wird dieser bei Android nicht nur
	durch Google, sondern auch durch die Open-Source Gemeinschaft fortwährend
	vorangetrieben. Auf der Seite von iOS hingegen geht die treibende Kraft in
	erster Linie von Apple aus. Allerdings existiert auch hier eine
	Nutzergemeinschaft welche indirekt an Neuerungen und Verbesserungen mitwirkt.
	Zuletzt gilt es zu erwähnen, dass viele der Sicherheitslücken durch die
	Nutzergemeinden aufgedeckt wurden. Das Beheben dieser hat oft zu
	sicherheitsrelevanten Optimierungen der mobilen Betriebssysteme geführt.

	%Referenzen zum Ende
	%bibtex referenzen
	\newpage
	\bibliography{general/bibtex/bibtex.bib}
	%Bibliography im Inhaltsverzeichnis anzeigen(muss unterhalb von bibliography
	% sein)
	\addcontentsline{toc}{section}{Literatur}
\end{document}
