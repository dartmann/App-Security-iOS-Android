\section{Schlusswort}
	Auf den ersten Blick erscheinen Android und iOS wie absolut differente mobile
	Betriebssysteme. Bei näherem Hinsehen wird aber schnell ersichtlich, dass diese
	in vielen Bereichen den selben Ansätzen folgen. Dabei liegt es in der Natur der
	Sache, dass es immer Sicherheitsmängel geben wird, da der Mensch an sich Fehler macht. 
	Ein stetiger Wandel bestimmt beide Platformen, welcher auch
	sicherheitstechnische Aspekte betrifft. Dabei wird dieser bei Android nicht nur
	durch Google, sondern auch durch die Open-Source Gemeinschaft fortwährend
	vorangetrieben. Auf der Seite von iOS hingegen geht die treibende Kraft in
	erster Linie von Apple aus. Allerdings existiert auch hier eine
	Nutzergemeinschaft welche indirekt an Neuerungen und Verbesserungen mitwirkt.
	Zuletzt gilt es zu erwähnen, dass viele der Sicherheitslücken durch die
	Nutzergemeinden aufgedeckt wurden. Das Beheben dieser hat oft zu
	sicherheitsrelevanten Optimierungen der mobilen Betriebssysteme geführt.