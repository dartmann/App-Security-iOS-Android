\section{Vorwort}
	Das Smartphone ist in unserer heutigen Welt nicht mehr wegzudenken. Es dient
	als Alltagshelfer mit vielen Funktionen. Es kann als Notizbuch, oder
	Terminkalender genutzt werden. Es ist eines der Hauptkommunikationsmittel, im
	gesprochenen, als auch geschriebenen Wort. Ebenso wird es auch zum
	Zeitvertreib oder für die Navigation genutzt. Mehrere Studien haben offen 
	gelegt, dass das Smartphone den Computer beziehungsweise Laptop - zumindest
	bei der Generation unter 18 Jahren - schlägt
	\cite{BitkomStudieJugend2014}
	\cite{MPFSStudie2013}.
	Dabei kann leicht vergessen werden, dass Smartphones mittlerweile die Leistung
	eines Bürocomputers haben und die Betriebssysteme dieser Geräte denen von PC
	und Laptop immer ähnlicher werden. Daraus ergeben sich für mobile Geräte
	ähnliche Gefahren wie für Computer und Laptop. Ein einfacher Vergleich:
	Anti-Viren Software ist auf den klassischen Computersystemen weit verbreitet,
	aber wer nutzt eine solche Software auf seinem Smartphone oder Tablet? Dabei
	sind auch auf diesen Geräten hoch sensible Daten gespeichert.\\
	Mit dieser Arbeit soll die Sicherheitsarchitektur der mobilen Betriebsysteme
	iOS und Android vorgestellt und erläutert werden. Weiterhin werden die
	Fähigkeiten beider Systeme in Bezug auf Gewährleistung dieser Sicherheit,
	sowohl auf Betriebsystem- als auch Applikationsebene beleuchtet.
