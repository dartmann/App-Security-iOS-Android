\section{Vorwort}
	Das Smartphone ist in unserer heutigen Welt nicht mehr wegzudenken. Es dient
	als Alltagshelfer mit vielen Funktionen. Es kann als Notizbuch, oder
	Terminkalender genutzt werden. Es ist eines der Hauptkommunikationsmittel, im
	gesprochenen, als auch geschriebenen Wort. Ebenso wird es auch zum
	Zeitvertreib, oder für die Navigation genutzt. Mehrere Studien haben offen 
	gelegt, dass das Smartphone den Computer bzw. Laptop - zumindest bei der
	Generation unter 18 Jahren - schlägt.
	\cite{BitkomStudieJugend2014}
	\cite{MPFSStudie2013}
	Viele vergessen dabei, dass Smartphones mittlerweile die Leistung eines kleinen
	Computers haben, die Betriebssysteme dieser Geräte denen von PC und Laptop
	immer ähnlicher werden und somit auch den selben Gefahren wie am PC ausgesetzt
	sind. Ein einfacher Vergleich, bietet hier schnelle	Klärung: Die hohe
	Verbreitung von Anti-Viren Software auf Desktop Rechnern oder Laptops ist
	selbstvertändlich, aber wer nutzt eine solche Software auf seinem
	Smartphone oder Tablet? Dabei hat man auch auf diesen Geräten hoch sensible
	Daten gespeichert.\\
	Aufgrund dessen wollen die Authoren mit dieser Arbeit auf
	sicherheitstechnische Aspekte und die Sicherheitsarchitektur von Android und iOS eingehen und
	Problematiken dieses Betriebssystems beleuchten, sowie aufzeigen welche
	Hilfsmittel diese für Entwickler und Nutzer bereitstellen.
