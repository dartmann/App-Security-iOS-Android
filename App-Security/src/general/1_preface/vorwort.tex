\section{Vorwort}
	Das Smartphone ist in unserer heutigen Welt nicht mehr wegzudenken. Es dient
	als Alltagshelfer mit vielen Funktionen. Es kann als Notizbuch, oder
	Terminkalender genutzt werden. Es ist eines der Hauptkommunikationsmittel, im
	gesprochenen, als auch geschriebenen Wort. Ebenso wird es auch zum
	Zeitvertreib, oder für die Navigation genutzt. Dabei begleitet es uns jeden
	Tag fast überall.
	Mehrere Studien haben offen gelegt, dass das Smartphone den Computer bzw. Laptop - zumindest bei der
	Generation unter 18 Jahren - schlägt.
	\footnote{http://www.bitkom.org/files/documents/BITKOM\_PK\_Kinder\_und\_Jugend\_3\_0.pdf}
	\footnote{http://www.mpfs.de/fileadmin/JIM-pdf13/JIMStudie2013.pdf}
	Das Smartphone hat in den letzten Jahren sich in den Alltag der meisten
	Menschen eingefügt (Statistik?!?!).  Es dient als Alltagshelfer, Notizbuch, 
	Terminkalender, Kommunikationsmittel und Zeitvertreib, und ist dabei fast überall dabei.\\
	Viele vergessen dabei, dass Smartphones mittlerweile die Leistung eines kleinen
	Computers haben und somit auch die selben Gefahren wie am PC Zuhause vorhanden sind.
	Kaum einer hat auf seinem PC kein Anti-Viren System installiert. Aber wer hat
	eines auf seinem Smartphone oder Tablet? Dabei hat man gerade auf diesen Geräten zum Teil 
	hoch sensible Daten gespeichert.\\
	Daher wollen wir im Folgenden auf sicherheitstechnische Aspekte des Android und
	iOS Betriebssystems eingehen und aufzeigen welche Hilfsmittel beide für
	Entwickler und Nutzer bereitstellen.
