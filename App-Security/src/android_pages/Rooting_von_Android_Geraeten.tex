\section{Rooting von Android-Geräten}

	Es gibt mittlerweile verschiedenste Android Versionen, die zum Teil auch abseits der Smartphonehersteller entwickelt wurden. Aber auch bei den 'standard' Versionen gibt es viele Möglichkeiten Android zu personalisieren und zu modifizieren. Dabei können Modifikationen soweit gehen, dass ganze Systemapps ersetzt werden. Hierfür sind aber oftmals Root-Rechte von Nöten. Dies ist zwar möglich, allerdings basiert ein Großteil der Sicherheitsarchitektur darauf, dass keine Nutzerapp Root-Rechte bekommt, wodurch ein Rooting schnell ein Sicherheitsrisiko darstellen kann. Schließlich kann der Root-Nutzer auf alle Daten auf dem Gerät zugreifen.
	Ein Beispiel wie ein solcher Zugriff mittels Exploit gelingen kann, ist in der Sektion Android: Sicherheit durch Open-Source zu finden. \ref{sec:Android_Sicherheit_Open_Source}
	