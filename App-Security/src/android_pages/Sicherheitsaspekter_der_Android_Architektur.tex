\section{Sicherheitsaspekte der Android-Architektur}

	Bereits durch die Architektur des Betriebssystems, insbesondere durch die restriktive Rechtevergabe und das Sandboxing, wird versucht ein möglichst sicheres System bereitzustellen.

	\subsection{Basis Rechtesystem}\label{sec:BasisRechteSystem}
	Von Linux wurde auch das Basis-Rechtesystem übernommen. Hierbei bekommt jede App eine eindeutige User-ID (UID) zugewiesen, welche im Normalfall zur Installationszeit zugeteilt wird. Jeder Nutzer, und somit auch jede App, arbeitet grundsätzlich erst einmal nur innerhalb der ihm zugewiesenen virtuellen Maschine und dem damit verbundenen Dateisystem.\\\\
	Da es dennoch in vielen Fällen nötig ist Daten zwischen verschiedenen Apps auszutauschen, gibt es mehrer Möglichkeiten dies zu tun. Die üblichen Wege wären Intends oder SharedPreferences. Zusätzlich gibt es noch die Möglichkeit mehreren Apps dieselbe UID zuweisen zu lassen. Dies ist allerdings nur möglich wenn die entsprechenden Applikationen mit dem selben Zertifikat signiert wurden und in deren Manifest Datei eine gemeinsame UID festgelegt wurde.
	Durch dieses Rechtesystem wird versucht sicherzustellen, dass kein Nutzerprogramm als \textit{root} ausgeführt wird.
	
	\subsection{Sandboxing und Permissions} \label{sec:SandBoxingNPermissions}