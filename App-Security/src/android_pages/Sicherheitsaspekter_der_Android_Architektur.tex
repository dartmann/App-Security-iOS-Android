\section{Sicherheitsaspekte der Android-Architektur}

	Bereits durch die Architektur des Betriebssystems, insbesondere durch die restriktive Rechtevergabe und das Sandboxing, wird versucht, ein möglichst sicheres System bereitzustellen. Unterstützt wird das Vorhaben ab Android Version 4.3 zusätzlich durch den Einsatz von Security-Enhanced Linux (SELinux).

	\subsection{Basis Rechtesystem}\label{sec:BasisRechteSystem}
	Von Linux wurde auch das Basis-Rechtesystem "ubernommen. Hierbei bekommt jede App eine eindeutige User-ID (UID) zugewiesen, welche im Normalfall zur Installationszeit zugeteilt wird. Jeder Nutzer, und somit auch jede App, arbeitet grundsätzlich erst einmal nur innerhalb der ihm zugewiesenen virtuellen Maschine und dem damit verbundenen Dateisystem.\\\\
	Da es dennoch in vielen Fällen nötig ist, Daten zwischen verschiedenen Apps auszutauschen, gibt es mehrer Möglichkeiten dies zu tun. Die üblichen Wege wären Intends oder SharedPreferences. Zusätzlich gibt es noch die Möglichkeit mehreren Apps dieselbe UID zuweisen zu lassen. Dies ist allerdings nur möglich, wenn die entsprechenden Applikationen mit dem selben Zertifikat signiert wurden und in deren Manifest Datei eine gemeinsame UID festgelegt wurde.
	Durch dieses Rechtesystem wird versucht sicherzustellen, dass kein Nutzerprogramm als \textit{root} ausgeführt wird.
	
	\subsection{Sandboxing und Permissions} \label{sec:SandBoxingNPermissions}
	Wie bereits erwähnt, laufen die Applikationen jeweils in ihrer eigenen Sandbox. Grundsätzlich ist die App damit in ihrer Ausführung auf ihren Bereich beschränkt und kann nicht mit anderen Prozessen und Daten ausserhalb interagieren. Dennoch ist es in den meisten Fällen sinnvoll mit Systemservices und Nutzerdaten zu interagieren, die nicht in der eigenen Sandbox verfügbar sind. 
	
	\subsubsection{Permissions im Detail}
	Um nun die bestehenden Zugriffsrechte erweitern zu können, müssen die entsprechenden Rechte (Permissions) in der Manifest Datei deklariert und angefordert werden. Zu Installationszeit werden diese Permissions dem Nutzer angezeigt und dieser wird gefragt, ob er den Rechtswünschen der App zustimmt oder nicht. Dabei gilt das \textit{Alles-Oder-Nichts-Prinzip}, d.h. entweder bekommt die Anwendung alle Rechte oder keine - was eine nicht Installation zur Folge hat. Des weiteren können die Berechtigungen nach der Installation nicht mehr angepasst werden.\\
	Oben genannte Berechtigungen sind beispielsweiße für Zugriffe auf externe Speichermedien oder auch die Kamera nötig. Dabei ist allerdings zu beachten, dass die Permissions zum Teil sehr grob definiert sind. Wodurch für den Nutzer nicht unbedingt erkenntlich ist, welche Informationen eine App warum abgreift und ob die App wirklich Gebrauch des Rechts macht.\\
	Anhand der folgenden Permission lässt sich die daraus resultierende Problematik gut erkennen:\\
	RECORD\_AUDIO Permission:
	\begin{quote}
	Allows an application to record audio \footnote{https://developer.android.com/reference/android/Manifest.permission.html\#RECORD\_AUDIO}
	\end{quote} 
	Dabei ist für den Nutzer nicht sichtbar, wann eine Aufnahme läuft, ausser die Applikation stellt dafür einen Hinweis bereit - wobei hier die Frage ist ob dieser auch wirklich verlässlich ist. Stellt die App einen Service bereit, kann ein solcher Mitschnitt auch im Hintergrund geschehen, und damit auch während eines Telefonats. Die einzige Chance RECORD\_AUDIO zur Laufzeit zu unterbinden ist, den Service bzw. die App über den Anwendungsmanager zu beenden.\\
	Dieses Problem ist auf die meisten anderen Berechtigungen übertragbar.\\
	Für die nächste Android Version, Android M, ist eine Verbesserung dieses Berechtigungssystems geplant und auch bereits vorgestellt worden. Es wird nun ermöglicht zur Laufzeit von Applikationen diesen Berechtigungen zu entziehen und freizugeben. Damit bekommt der Nutzer deutlich mehr Möglichkeiten, um seine Daten zu schützen. Allerdings wird es wohl kein Update für ältere Versionen geben, wodurch dort das Problem bestehen bleibt.
	
	\subsubsection{Besonderheit: Systemapps}
	Apps des Smartphone Hersteller können Rechte besitzen, die für normale Anwendungen nicht verfügbar sind, um Basis Apps und Services bereit zustellen. Hierfür werden alle Hersteller Applikationen mit sogenannten \textit{Publisher Keys} signiert, dadurch können, wie bereits unter \ref{sec:BasisRechteSystem} erklärt, diese Anwendung bspw. im selben Prozess laufen.
	
	\subsection{SELinx in Android}
	\begin{quote}
	SELinux operates on the ethos of default denial. Anything that is not explicitly allowed is denied.\cite{SELinuxAndroid}
	\end{quote}
	Durch die Nutzung von SELinux soll das vorhandene Berechtigungssystem und Sandboxing unterstützt und verstärkt werden. Dabei sind zwei Nutzungsmodi zu unterscheiden. Während im \textit{permissive mode} Regelverstöße nur geloggt werden, wird im \textit{enforcing mode} die strikte Einhaltung erzwungen. In den Versionen 4.3 bis exklusive 5.0 war der \textit{enforcing mode} nicht überall in Nutzung. Dies änderte sich ab Version 5.0, seit dem läuft nur noch dieser Modus.
	 
	
	

	