\section{Sicherheitsaspekte der Android-Architektur}

	Bereits durch die Architektur des Betriebssystems, insbesondere durch die restriktive Rechtevergabe und das Sandboxing, wird versucht ein möglichst sicheres System bereitzustellen.

	\subsection{Basis Rechtesystem}\label{sec:BasisRechteSystem}
	Von Linux wurde auch das Basis-Rechtesystem "ubernommen. Hierbei bekommt jede App eine eindeutige User-ID (UID) zugewiesen, welche im Normalfall zur Installationszeit zugeteilt wird. Jeder Nutzer, und somit auch jede App, arbeitet grunds"atzlich erst einmal nur innerhalb der ihm zugewiesenen virtuellen Maschine und dem damit verbundenen Dateisystem.\\\\
	Da es dennoch in vielen F"allen n"otig ist Daten zwischen verschiedenen Apps auszutauschen, gibt es mehrer M"oglichkeiten dies zu tun. Die "ublichen Wege w"aren Intends oder SharedPreferences. Zus"atzlich gibt es noch die M"oglichkeit mehreren Apps dieselbe UID zuweisen zu lassen. Dies ist allerdings nur m"oglich wenn die entsprechenden Applikationen mit dem selben Zertifikat signiert wurden und in deren Manifest Datei eine gemeinsame UID festgelegt wurde.
	Durch dieses Rechtesystem wird versucht sicherzustellen, dass kein Nutzerprogramm als \textit{root} ausgeführt wird.
	
	\subsection{Sandboxing und Permissions} \label{sec:SandBoxingNPermissions}