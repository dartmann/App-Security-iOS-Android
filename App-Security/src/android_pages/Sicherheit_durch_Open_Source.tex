\section{Android: Sicherheit durch Open-Source}
Das Android Betriebssystem wird in erster Linie von Google und dem Android Open Source Projekt gewartet und weiterentwickelt. Durch dieses Vorgehen erhofft man sich eine stetige Verbesserung des Systems, sowohl aus funktionaler, wie auch sicherheitstechnischer Sicht. \\
Durch ein öffentliches Bereitstellen des Source Codes soll auch die Wahrscheinlichkeit für Sicherheitslücken minimiert werden, da somit mehr Entwickler die Möglichkeit bekommen sich diesen anzuschauen und Sicherheitslücken zu entdecken. Dass dies allerdings nicht garantiert ist, wurde zuletzt durch den Heartbleed-Bug in der SSL Bibliothek OpenSSL bekannt. Obwohl OpenSSL Open-Source ist, und viele Entwickler daran arbeiten, konnte die Schwachstelle über längere Zeit hinweg unentdeckt bleiben. Dennoch wird es, durch das öffentliche Bereitstellen des Programmcodes, deutlich schwerer versteckte Programmierschnittstellen mit einzubauen, wie es beispielsweiße in iOS der Fall ist.\\
Ein großes Plus ist, dass dadurch, dass ein Linux-Kernel die Basis von Android ist, die stetige Entwicklung des Kernels sich auch in die Entwicklung des Android Betriebssystems niederschlägt und umgedreht.\\
Allerdings darf nicht vergessen werden, dass die meisten Smartphone Hersteller Android für ihre Geräte erst noch modifizieren und Closed-Source Apps mit erweiterten Rechten installieren können. Solche Modifikationen sind zumeist nicht Open-Source. Zum Beispiel sind in dem offiziellen Android Release von Google, immer einige Google eigene Apps vorinstalliert, deren Source Code nicht einsehbar ist. So ist mittlerweile bekannt geworden, dass über die App \textit{Google Einstellungen} (engl. \textit{Google Settings}) Google in der Lage ist, per Fernzugriff Applikationen zu löschen und zu installieren. Dies wurde bekannt, nachdem eine App eines Forschers aus dem App Market (heute Play Store) entfernt wurde und die Funktion zum Löschen der Apps genutzt wurde um restliche Installationen zu entfernen.\footnote{http://android-developers.blogspot.de/2010/06/exercising-our-remote-application.html} 
Diese Funktion ist angedacht, um Schadcode von infizierten Geräten zu entfernen, kann aber auch genutzt werden um Apps zu installieren.\footnote{http://www.computerworld.com/article/2506557/security0/google-throws--kill-switch--on-android-phones.html} Aktuell ist nicht bekannt, was über diese App noch steuerbar ist.\\
Bei solchen Funktionen stellt sich immer die Vertrauensfrage. Der eigentliche Sinn und Zweck bringt Komfort und schnelle Abhilfe. Die Frage ist nur, wie gut solche Funktionen abgesichert sind und ob sie für andere Zwecke von dritter Missbraucht werden können. Selbst wenn derartige Applikationen gegen den unberechtigten Zugriff dritter abgesichert ist, bleibt noch offen ob staatliche Institutionen von Firmen wie Google Zugang bekommen oder erzwingen können. Wie groß die Angst vor derartigen Zugriffen ist, kann man an der seit gut zwei Jahren laufenden Überwachungsaffäre, welche durch Edward Snowden angestoßen wurde, erkennen.