\section{Android: Sicherheit durch Open-Source} \label{sec:Android_Sicherheit_Open_Source}
Das Android Betriebssystem wird in erster Linie von Google und dem Android Open-Source Projekt gewartet und weiterentwickelt. Durch dieses Vorgehen erhofft man sich eine stetige Verbesserung des Systems. \\
Mit dem öffentlichen Bereitstellen von Source Code geht oft die Hoffnung einher, dass durch das Mehr-Augen-Prinzip Fehler im Programmcode schneller gefunden und behoben werden können - insbesondere auch Sicherheitslücken.
%Durch ein öffentliches Bereitstellen des Source Codes soll auch die Wahrscheinlichkeit für Sicherheitslücken minimiert werden, da somit mehr Entwickler die Möglichkeit bekommen sich diesen anzuschauen und Sicherheitslücken zu entdecken.
Dass dies allerdings nicht garantiert ist, wurde zuletzt durch den Heartbleed-Bug in der SSL Bibliothek OpenSSL bekannt. Obwohl OpenSSL Open-Source ist, und viele Entwickler daran arbeiten, konnte die Schwachstelle über längere Zeit hinweg unentdeckt bleiben. Dennoch wird es, durch das öffentliche Bereitstellen des Programmcodes, deutlich schwerer versteckte Programmierschnittstellen mit einzubauen, wie es beispielsweiße in iOS der Fall ist (vgl.\ref{sec:undocumented-services}).\\
Ein großes Plus ist, dass dadurch, dass ein Linux-Kernel die Basis von Android ist, die stetige Entwicklung des Kernels sich auch in die Entwicklung des Android Betriebssystems niederschlägt und umgedreht.\\
Trotz dass Android open-source ist, darf nicht vergessen werden, dass die Smartphone Hersteller das System noch für die eigenen Geräte modifziert und der veränderte Programmcode nicht immer frei zugänglich ist. In der Vergangenheit kam es auf dieses Weise schon zu Sicherheitslücken, die nur in den Android Versionen einzelner Hersteller zu finden waren. Beispiel:\\
Von einem Nutzer des XDA-Developer Forums wurde im Dezember 2012 eine Sicherheitslücke entdeckt, die es ermöglicht auf mehreren Samsung Geräten an Root-Rechte heranzukommen. Einschränkung dieser Sicherheitslücke waren die nutzung des Samsung Kernels in Zusammenarbeit mit einem von zwei Prozessoren von Exynos. Basis dieses Exploits war, dass jeder Nutzer vollständigen Zugriff auf den physikalischen Speicher hatte, was unter anderem genutzt werden konnte um eine Root-Shell zu öffnen.\footnote{http://forum.xda-developers.com/showthread.php?p=35469999} \\
Auch können von Herstellern Apps installiert werden, welche mit einem Zertifikat des Herstellers signiert sind. Ein solches Vorgehen kann, wie unter \ref*{sec:SandBoxingNPermissions} beschrieben, zu erhöhten Rechten führen.
So sind beispielsweiße in dem offiziellen Android Release von Google, immer einige Google eigene Apps vorinstalliert, deren Source Code nicht einsehbar ist. Mittlerweile wurde bekannt, dass über die App \textit{Google Einstellungen} (engl. \textit{Google Settings}) das Unternehmen in der Lage ist, per Fernzugriff Applikationen zu löschen und zu installieren. Bekannt wurde dies, nachdem die App eines Forschers aus dem App Market (heute Play Store) entfernt wurde und die Funktion zum Löschen der \textit{Google Einstellungen} genutzt wurde um restliche Installationen zu entfernen.\footnote{http://android-developers.blogspot.de/2010/06/exercising-our-remote-application.html} 
Diese Funktion ist angedacht, um Schadcode von infizierten Geräten zu entfernen, kann aber auch genutzt werden um Apps zu installieren.\footnote{http://www.computerworld.com/article/2506557/security0/google-throws--kill-switch--on-android-phones.html} Aktuell ist nicht bekannt, was über diese App noch steuerbar ist.\\
Bei solchen Funktionen stellt sich immer die Vertrauensfrage. Der eigentliche Sinn und Zweck bringt Komfort und schnelle Abhilfe. Die Frage ist nur, wie gut solche Funktionen abgesichert sind und ob sie für andere Zwecke von dritter Missbraucht werden können. Selbst wenn derartige Applikationen gegen den unberechtigten Zugriff dritter abgesichert ist, bleibt noch offen ob staatliche Institutionen von Firmen wie Google Zugang bekommen oder erzwingen können. Wie groß die Angst vor derartigen Zugriffen ist, kann man an der seit gut zwei Jahren laufenden Überwachungsaffäre, welche durch Edward Snowden angestoßen wurde, erkennen.

\subsection{Update-Problematik bei Android}
Wie bereits in vorherigen Sektionen erwähnt, wird das Betriebssystem von den Smartphone Herstellern vor dem Vertrieb zumeist noch angepasst. Dieses Vorgehen führt auch dazu, dass sämtliche Updates über die Hersteller gehen müssen. Dadurch kann es vorkommen, dass es durchaus etwas länger dauert bis ein solches Update wirklich auf allen Android Geräten zu finden ist - wenn das Update überhaupt verteilt wird. Gerade bei Updates älterer Android Versionen wird oftmals darauf verzichtet dieses bereitzustellen. Davon betroffen sind allerdings nicht nur Updates welche die Benutzerführung oder ähnliches betreffen, sondern auch solche, die eine Sicherheitslücke schließen, oder allgemein die Sicherheit auf den Geräten Verbessern würden. Hinzukommt auch, dass Android Versionen heute schnell als veraltet gelten, und daher immer öfter grundsätzlich die Entwicklung von Updates eingestellt wird.\cite{Drake2014}\\
Diese Problematik wird auch nicht durch ein öffentliches Bereitstellen, des grundlegenden Programmcodes gelöst.

\subsection{Transparenzprobleme bei Android}
Dadurch, dass Smartphone Hersteller und andere, Modifikationen am Android Betriebssystem vornehmen, ist es für den Endnutzer nicht immer leicht, festzustellen ob gewisse Änderungen sicherheitstechnische Auswirkungen haben. So könnten Sicherheitsvorkehrungen wie beispielsweise die Boot-Verifikation deaktiviert sein, ohne das der Nutzer das merkt.