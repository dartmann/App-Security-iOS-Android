\subsection{Android}
	\subsubsection{Verschlüsselung von Datenträgern}
	Mit Android 3.0 (Honeycomb) wurde die Möglichkeit eingeführt, die userdata-Partition vollständig zu verschlüsseln (Fulldisk Encryption - FDE). Basis der Verschlüsselung ist, wie bei der Verifikation der Partitionen (\ref{sec:VerifikationDerBootmedien}), eine Funktion des Device Mappers - dm-crypt.\\\\
	Zum Ent-/Verschlüsseln wird der Blockchiffrenmodus \textit{Cipher Block Chaining (CBC)} mit einem zufällig generierten 128-Bit AES Schlüssel (Disk Encryption Key, \textit{DEK}) verwendet. Die Verschlüsselung erfolgt Sektorweise. Da nicht immer seriell aus dem Speicher gelesen wird, kann nicht nur ein Initialisationsvektor (\textit{IV}) für die komplette Ent- und Verschlüsselung genutzt werden. Daher wird stattdessen für jeden Sektor anhand eines, vom Disk Encryption Key abgeleiteten, Salts \textit{S} und der Sektornummer \textit{SN} ein eigener IV berechnet.\\
	Somit gilt:
\begin{center}
	\begin{math}
	IV(SN) = AES_{S}(SN)\end{math}, mit \begin{math}S = SHA256(DEK)
	\end{math}
\end{center}
\chapter{test}
	Diese Art des Berechnung für einen IV nennt sich \textit{Encrypted Salt-Sector Initialization Vector} unter Nutzung der Hashfunktion \textit{SHA256} (\textit{ESSIV:SHA256}).\cite[S. 259]{Elenkov2014} Das Nutzen dieses Verfahren in dieser Art und Weise schützt die Daten allerdings nicht vor Manipulationen, da keinerlei Integritätscheck vorgenommen wird. Es wurde bereits demonstriert, dass es möglich ist, in ein verschlüsseltes Ubuntu 12.04 eine Backdoor einzuschleusen. Da das von Ubuntu genutzte Verschlüsselungsverfahren zu dem von Android identisch ist, lässt sich dieser Angriff übertragen. \footnote{http://www.jakoblell.com/blog/2013/12/22/practical-malleability-attack-against-cbc-encrypted-luks-partitions}\\
	Um den Disk Encryption Key zu sichern, wird dieser mit einem 128-Bit AES Key Encryption Key (KEK) verschlüsselt. Der KEK wird von einem, durch den Nutzer bestimmtes, Passwort abgeleitet. In der Vergangenheit, bis einschließlich zur Version 4.3, wurde für die Ableitung der Algorithmus \textit{PBKDF2} mit 2.000 Iterationen genutzt. Zusätzlich wurde noch ein 128-Bit Salt verwendet, welcher einem Zufallsgenerator entstammt. Da PBKDF2 nicht mehr aktuellen Sicherheitsstandards entspricht, wird seit Android 4.4 \textit{scrypt} als Ableitungsfunktion genutzt. Der DEK wird in verschlüsselter Form, zusammen mit dem Salt, in den Metadaten der verschlüsselten Partition abgelegt. Um eine Veränderung des DEK zu verhindern, bzw. die Integrität des DEK feststellen zu können, wird ab Version 5.0 dieser mit einem Key (Hardware-Bound Key, HBK) aus der TEE signiert. \\\\
	Damit ergibt sich folgendes Vorgehen:
	\begin{quote}
		\begin{enumerate}
		   \item Generate random 16-byte disk encryption key (DEK) and 16-byte salt.
		   \item Apply scrypt to the user password and the salt to produce 32-byte intermediate key 1 (IK1).
		   \item Pad IK1 with zero bytes to the size of the hardware-bound private key (HBK). Specifically, we pad as: 00 || IK1 || 00..00; one zero byte, 32 IK1 bytes, 223 zero bytes.
		   \item Sign padded IK1 with HBK to produce 256-byte IK2.
		   \item Apply scrypt to IK2 and salt (same salt as step 2) to produce 32-byte IK3.
		   \item Use the first 16 bytes of IK3 as KEK and the last 16 bytes as IV.
		   \item Encrypt DEK with AES\_CBC, with key KEK, and initialization vector IV. 
	   \end{enumerate}
	   \footnote{https://source.android.com/devices/tech/security/encryption/index.html}
	\end{quote}
	Als Nutzerpasswort kann eine Pin oder ein klassisches Passwort dienen. Seit Version 5.0 ist es auch möglich ein Eingabemuster als Passwort zu nutzen. \\
	Ein Vorteil, der durch das nutzen eines KEK und eines DEK zustande kommt, ist der, dass bei einer Änderung des Nutzerpasswortes nicht die komplette Partition neu verschlüsselt werden muss, sondern lediglich der DEK. Android 5.0 Geräte werden mit dem ersten Start automatisch, unter Nutzung eines Standard Passwortes, verschlüsselt. 
	
	\subsubsection{Keystore Service}
	Die FDE greift auf Partitions- und damit auf der Geräteebene. Hat es Schadcode erst einmal in ein laufendes System geschafft, wird dadurch der Schutz aber nicht mehr sichergestellt. Um dennoch einzelne Daten auf der Anwendungsebene sicher speichern zu können, kann mithilfe der durch das Entwicklungsfr amework bereitgestellte \textit{Cryptography Service Provider (CSP)} zurückgreifen. CSP stellen Algorithmen und Bibliotheken zum ent-/verschlüsseln, sowie zur Sicherstellung der Integrität von Daten bereit. Sollten die CSP aus dem Android Framework nicht reichen, kann man externe Nachladen. Dabei sollte man darauf achten, dass diese aus vertrauenswürdigen Quellen stammen, da die Sicherheit stark von der Richtigkeit der eingesetzten Algorithmen abhängig ist. Übrig bleibt dabei noch die Problematik, wie die Schlüssel für die Verschlüsselungen gespeichert werden sollen. Hierfür ist, seit Android 1.6, im Kernel ein Credential Service implementiert. Die im Service gespeicherten Daten werden wiederum durch ein vom Endnutzer bereitgestelltes Password verschlüsselt. Das Passwort kann in Form eines Musters, PINs oder eines üblichen Text Passworts vorliegen. Jeder Nutzer hat nur Zugriff auf die Daten im Keystore, die auch von ihm angelegt wurden.\\\\	
	Bis zur Android Version 4.0 wurden in diesem Keystore jedoch nur Netzwerkschlüssel abgelegt werden. Mit Version 4.3 wurde als zusätzlichen Schutz hardwareseitige Unterstützung mithilfe der TEE eingebaut.
