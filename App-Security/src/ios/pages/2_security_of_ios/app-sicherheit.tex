\section{Applikationssicherheit unter iOS}
	Mobile Betriebssysteme bauen zum größten Teil auf Applikationen - weiter
	\textsl{Apps} genannt - auf, welche die produktive Nutzung eines mobilen
	Endgerätes erheblich verbessern können. Dabei ist es allerdings
	essentiell, dass diese Apps korrekt vom Betriebssystem behandelt werden, da
	andernfalls die Systemsicherheit, die Stabilität oder gar die Nutzerdaten
	gefährdet werden können. Wie bereits im Kapitel Systemsicherheit (siehe:
	\ref{sec:components-syssec}) vorgestellt, wird auch hier eine Art
	Schichtensystem angewendet, um eine Signierung und Verifikation, sowie ein
	Sandboxing der Apps sicherzustellen.
	\subsection{Signieren von Applikationen}
		Nach dem Start des iOS Kernels, stellt dieser sicher, welche Nutzerprozesse
		und Apps gestartet werden dürfen. Dazu werden diese auf eine Signierung durch
		ein von Apple ausgestelltes Zerifikat geprüft. Das zwingende Vorhandensein
		dieses Zertifikats stellt eine Adaption der \textsl{chain of trust} (siehe:
		\ref{sec:secure-boot-chain}) auf die Applikationsebene dar. So muss jeder
		private Entwickler als auch jedes Unternehmen seine Identität bei Apple
		verifizieren, bevor ein Entwicklerzertifikat von Apple für diese ausgestellt
		wird. Somit ist sicher gestellt, dass jede App im AppStore auf eine
		Privatperson zurückzuführen ist, was auch ein gesteigertes Vertrauen der
		Nutzer in die Qualität der Apps zur Folge hat.\\
		Allerdings muss an diesem Punkt erwähnt werden, dass Apple Ausnahmen dieser
		Verifikation in Form des \textsl{iOS Developer Enterprise
		Program}\footnote{https://developer.apple.com/programs/ios/enterprise/}
		erlaubt und so Apps auch vorbei am AppStore und auf iOS Geräte installiert
		werden können. Dabei prüft Apple das anfragende Unternehmen auf Eignung durch
		deren D-U-N-S Nummer - einem Zahlensystem zur eindeutigen Identifizierung von
		Firmen. Populär wurde eine jüngste Ausnutzung dieses Privileges, bei welcher
		über eine Webseite bei Einwilligung eine App installiert wird, welche dann
		ein Abonnement verkaufen will\footnote{http://heise.de/-2679222}. Apple
		dachte diese Möglichkeit nur für Firmen, die ihr eigenes Mobile Device
		Management betreiben, an. Hier wurde also entweder der ausstellende Account
		gekapert oder diese Lizenz vom Eigentümer schlichtweg missbraucht.\\
		Ab iOS8 wird es Entwicklern erlaubt in ihren Apps Frameworks zu verwenden. Um
		hier ein Laden von unsigniertem Code zu verhindern, wird beim Start einer App
		die \textsl{Team-ID} geprüft - ein 10 stelliger alphanumerischer String,
		welcher aus dem von Apple ausgestellten Entwicklerzertifikat extrahiert wird.
		Eine App darf nur Code laden, welcher entweder vom System kommt, oder die selbe
		Team-ID besitzt.
	\subsection{}