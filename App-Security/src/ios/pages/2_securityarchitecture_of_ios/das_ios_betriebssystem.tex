\section{Das iOS Betriebsystem}
	Die aus Cupertino in Kalifornien stammende US-Amerikanische Apple Corporation
	ist eine der größten Firmen der Welt und hatte einen Umsatz von 182 Mrd.
	USD im Geschäftsjahr 2014. Sie ist ebenfalls der Erfinder des mobilen
	Betriebssystems iOS, welches auf den firmeneigenen Geräten iPad, iPad mini,
	iPhone, iPod touch und dem Apple TV ab der zweiten Generation zum Einsatz kommt.
	Der Kern des Betriebssystems basiert auf dem freien UNIX Betriebssystem Darwin,
	das auch als Vorlage für das Betriebssystem OS X genutzt wurde.\\
	Im Februar 2015 hatte iOS in den USA einen Marktanteil von 38,8\% und 17,4\%
	in Deutschland.
	%TODO: ask kristoffer, which way of presenting the links is his prefered way
	\footnote{http://www.kantarworldpanel.com/global/smartphone-os-market-share/}
	Welche Auswirkung hätte es, wenn Sicherheitslücken auf diesen zu finden wären,
	oder schlimmer noch, Hintertüren eingebaut sind, welche ohne das Wissen des
	Nutzers ausgenutzt werden könnten? Der iOS Teil dieser Arbeit soll unter
	anderem zeigen, dass ein Proprietäres Betriebssystem viele Vorteile mit sich
	bringt, aber auch gewisse Nachteile besitzt die es nicht zu verachten gilt.
	Ebenso werden historische Schwachstellen und Exploits behandelt und mit 
	praktischen Beispielen gezeigt, dass iOS - obwohl proprietär - dennoch
	angreifbar ist und Lücken aufweist.
