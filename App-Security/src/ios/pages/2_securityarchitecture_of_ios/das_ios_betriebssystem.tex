\section{Das iOS Betriebsystem}
	Die aus Cupertino in Kalifornien stammende US-Amerikanische Apple Corporation
	ist eine der größten Firmen der Welt und hatte einen Umsatz von 182 Mrd.
	USD im Geschäftsjahr 2014. Sie ist ebenfalls der Erfinder des mobilen
	Betriebssystems iOS, welches auf den firmeneigenen Geräten iPad, iPad mini,
	iPhone, iPod touch und dem Apple TV ab der zweiten Generation zum Einsatz kommt.
	Der Kern des Betriebssystems basiert auf dem freien UNIX Betriebssystem Darwin,
	das auch als Vorlage für das Betriebssystem OS X genutzt wurde.\\
	Im Februar 2015 hatte iOS in den USA einen Marktanteil von 38,8\% und 17,4\%
	in Deutschland.
	\footnote{http://www.kantarworldpanel.com/global/smartphone-os-market-share/}
	Welche Auswirkung hätte es, wenn Sicherheitslücken auf diesen zu finden wären,
	oder schlimmer noch, diese völlig ohne das Wissen des Nutzers ausgenutzt werden
	könnten? Mit dieser Arbeit möchte ich besonders auf die Sicherheitsarchitektur
	von iOS eingehen und gewisse Problematiken dieses Betriebssystems beleuchten,
	sowie zeigen, dass ein Proprietäres Betriebssystem viele Vorteile mit sich
	bringt, aber auch gewisse Nachteile besitzt die es nicht zu verachten gilt.
	Ebenso werde ich versuchen mit praktischen Beispielen zu zeigen, dass iOS -
	obwohl proprietär - dennoch angreifbar ist und Lücken aufweist.