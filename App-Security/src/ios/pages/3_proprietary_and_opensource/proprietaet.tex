\section{Proprietät von iOS}
	Definition Proprietär:
	\begin{quote}
		Synonym for closed-source or non-free, e.g. software issued without license
		rights permitting the public to independently review, develop and redistribute it.
		\footnote{http://www.catb.org/jargon/html/P/proprietary.html}
	\end{quote}
	Im Gegensatz zu anderen Herstellern mobiler Betriebssysteme lizenziert Apple
	iOS nicht für andere Hardwarehersteller, sondern produziert alles im
	eigenen Hause. IOS wird somit nur auf Hardware aus dem eigenen kontrollierten
	Herstellungsprozess eingesetzt. Dieser proprietäre Ansatz birgt gewisse
	Gefahren, die in diesem Kapitel (siehe: \ref{sec:undocumented-services})
	beleuchten werden sollen. Außerdem hat sich gezeigt, dass die Nutzergemeinde
	versucht aus der Proprietät zu entkommen bzw. sie zu umgehen (siehe:
	\ref{sec:jailbreaking}).
	
	\subsection{Undokumentierte und kritische Dienste}\label{sec:undocumented-services}
		%TODO: rewrite this header
		% Drei Dienste wurden erst im Juni 2013 offiziell dokumentiert, siehe:
		% https://support.apple.com/en-us/HT203034 und das auch erst nachdem sie
		% zdziarski in seinem paper gerügt hat.
		Apple kann SMS, Fotos, Videos, Kontakte, Musik, Aufnahmen und Anruferhistorie
		aus passcode geschützten Geräten auslesen. Möglich machen dies nicht
		dokumentierte Dienste, welche auf jedem Gerät mit iOS installiert sind. In
		diesem Kapitel will ich auf diese Dienste eingehen und deren genauen
		Einsatzzweck erläutern.
		\subsubsection{lockdownd - remote access}
			Der Dienst \textsl{lockdownd} ermöglicht den Zugriff auf ein iOS Gerät
			per TCP auf Port 62078 über einen USB-Anschluss. Dies wird über das eigene
			Protokoll \textsl{usbmux} gehandhabt. Dieses erlaubt TCP Verbindungen zum
			lokalen System über einen USB Anschluss mit übergebenen	Portnummern.
		\subsubsection{com.apple.mobile.pcapd}
			Pcapd stellt einen Überwachungsdienst dar, welcher auf der
			Programmierschnittstelle
			pcap\footnote{http://www.tcpdump.org/manpages/pcap.3pcap.html} aufbaut und
			diese in Form der Bibliothek \textsl{libpcap} implementiert. Mit diesem
			Dienst ist es möglich jeglichen Netzwerkverkehr mitzulesen und zu speichern.
			Sniffer\footnote{http://de.wikipedia.org/wiki/Sniffer}, z.B.
			Whireshark\footnote{https://www.wireshark.org/}, greifen auf diese
			Schnittstelle zurück, um Pakete direkt an der Netzwerkschnittstelle
			abzufragen. Dem Benutzer des iOS Gerätes wird es visuell nicht erläutert,
			wenn dieser Dienst aktiviert ist. Zusätzlich ist kein Entwickler-Modus von
			nöten, um diesen zu aktivieren. Seit iOS8 ist es nicht mehr möglich diesen
			Dienst über WiFi anzusprechen.
		\subsubsection{com.apple.mobile.file\_relay}
			Hinter diesem Paket verbirgt sich ein Dienst mit dem auf sensible und
			persönliche Daten zugegriffen werden kann. Darunter auch das Adressbuch, GPS
			Daten, Fotos und der Möglichkeit ein Image der Metadaten des Dateisystems
			abzugreifen. Apple schreibt dazu:
			\begin{quote}
				In iOS 8 and later, this capability requires additional configuration before
				use.
			\end{quote}
			Was eventuell auf eine Verbesserung des Zugriffs auf diesen Dienst hinweisen
			könnte. Beispielsweise nur noch Authorisierter vom Benutzer erlaubter
			Zugriff.
		\subsubsection{com.apple.mobile.house\_arrest}
			Apple dokumentiert hier eine Verwendung von iTunes für den Transfer von Daten
			auf und von dem Gerät. Zusätzlich wird dieses Feature von Xcode verwendet,
			wenn Testdaten zu einem Gerät geschickt werden sollen und sich die relevante
			App im Entwicklermodus befindet. Laut Zdziarski hat dieser Dienst Zugriff auf
			die Library, den Cache, Cookies und bevorzugte Ordner, welche hoch sensible
			Daten beinhalten - wie den Zwischenspeicher des sozialen Streams von Facebook
			oder Twitter - obwohl die iTunes GUI dies nicht erlaubt.
			
	\subsection{Historische Exploits}
		In der Geschichte von iOS hat es bereits diverse
		Exploits\footnote{http://wirtschaftslexikon.gabler.de/Definition/exploit.html}
		gegeben. In diesem Kapitel will ich einige der bekanntesten vorstellen.
		\subsubsection{libTiff Exploit} 
			Eine auf einem Pufferüberlauf der libtiff Bibliothek basierenden
			Sicherheitslücke. Dabei wurde die genannte Bibliothek gepatcht, um einen
			Jailbreak zu starten.
			\footnote{https://www.theiphonewiki.com/wiki/LibTiff\_Exploit} Dieser Exploit
			stammt aus der Anfangszeit von iOS in der alle Prozesse noch mit root
			Rechten liefen.
		\subsubsection{Ikee Virus}\label{sec:ikee-exploit}
			Dabei handelt es sich um einen der ersten Würmer unter iOS. Der	Author Ashley Towns
			wollte 2009 damit auf nicht veränderte Standardpasswörter der SSH Zugänge,
			bei iPhones mit Jailbreak hinweisen. Er nutzte mit dem in C geschriebenen
			Programm lediglich das Passwort \textsl{alpine}, deaktivierte bei
			erfolgreichem login den SSH Dienst und versuchte dann das Skript auf
			zufällig ausgewählten IP-Adressen auszuführen.
			\footnote{https://www.theiphonewiki.com/wiki/Ikee-virus}. Eine unter dem
			Namen \textsl{Ikee.B} veränderte Variante stahl Inhalte des Gerätes und
			machte dieses zum Zombie und somit Teil eines Botnetzes.
		\subsubsection{SpyPhone}
			Diese App wurde vom Author Seriot Nicolas\footnote{http://seriot.ch/} als
			\textsl{proof of concept} erstellt. Dabei nutzte er nur die gegebenen
			legitimen API's zum Erstellen von Apps für den AppStore. Obwohl diese App
			komplett innerhalb ihrer von iOS vorgeschriebenen Sandbox lief, konnte
			Sie dennoch:
			\begin{itemize}\itemsep0pt
				\item{Die Nummer des Gerätes auslesen}
				\item{Vom Adressbuch lesen und ebenfalls darin schreiben}
				\item{E-Mail Kontoinformationen auslesen}
				\item{Den Zwischenspeicher der Tastatur mitlesen}
				\item{Auf das GPS zugreifen}
				\item{WLAN Zugangspunkte auslesen}
			\end{itemize}
			
	\subsection{Jailbreaking}\label{sec:jailbreaking}
		Der aus dem englischsprachligen Raum stammende Begriff bezeichnet unter iOS
		das gezielte Aushebeln der von Apple vorgegebenen Restriktionen. Oft werden
		dabei eigene Paketverwaltungstools wie
		Cydia\footnote{http://cydia.saurik.com/} installiert, um auch von Apple nicht
		signierte Apps auf die modifizierten Geräte aufspielen zu können.\\
		Es werden zwei Arten von Jailbreaks unterschieden, der \textsl{tethered
		jailbreak} und der \textsl{untethered jailbreak}. Dabei bedeuted tethered,
		dass eine Anbindung an einen Computer, auf welchem eine Jailbreaking Software
		- beispielsweise \textsl{redsn0w}\footnote{http://blog.iphone-dev.org/}, oder
		\textsl{TaiG}\footnote{http://www.taig.com/en/} - installiert sein muss, die
		den Prozess unterstützt.
		Untethered Jailbreaks kommen ohne diese Anbindung aus und arbeiten autark.
		Das Prinzip des Jailbreakings ist dabei immer das selbe: es wird mit
		Hilfe eines Exploits das System kompromittiert. Zu Beginn wurden oft
		Hardwarefehler ausgenutzt, die während des
		DFU-Modus (siehe: \ref{sec:components-syssec}) installiert wurden.
		Später kamen \textsl{Userland-Exploits} dazu, welche während dem
		laufenden Betrieb von iOS gestartet werden und sich Schwachstellen im
		Betriebssystem zu nutze machen.
		%TODO: after dicussion with junker-schilling add more content or not
	
	\subsection{Vorteile der Proprietät}
		Obwohl der Nutzer unter iOS dazu gezwungen wird, sich den Vorgaben zu beugen,
		hat es auch seine Vorteile, dies zu akzeptieren. So sind bei aktivem Jailbreak
		Sicherheitsmechanismen wie die Signaturprüfung (siehe:
		\ref{sec:code-signing}), oder die Apple's Secure Boot
		Chain (siehe: \ref{sec:secure-boot-chain}) modifiziert beziehungsweise
		deaktiviert worden, was ein enormes Sicherheitsrisiko darstellen kann.
		Außerdem haben uns Exploits wie Ikee (siehe: \ref{sec:ikee-exploit}) gezeigt,
		dass zu viel Macht an falscher Stelle verbunden mit Unwissenheit meist mehr
		Schaden anrichtet, statt Erleichterung zu schaffen.
		Das BSI warnt zudem in einem Überblickspapier zu iOS offiziell vor dem Einsatz
		von Jailbreaks.\footnote{https://www.bsi.bund.de/SharedDocs/Downloads/DE/BSI/Grundschutz/Download/Ueberblickspapier\_Apple\_iOS\_pdf}
