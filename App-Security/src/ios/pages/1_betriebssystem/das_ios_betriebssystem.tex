\subsection{iOS}
	Die aus Cupertino in Kalifornien stammende US-Amerikanische Apple Corporation
	ist eine der größten Firmen der Welt und hatte einen Umsatz von 182 Milliarden
	US-Dollar im Geschäftsjahr 2014. Sie ist der Erfinder des mobilen
	Betriebssystems iOS, welches auf den firmeneigenen Geräten iPad, iPad mini,
	iPhone, iPod touch und dem Apple TV ab der zweiten Generation zum Einsatz
	kommt. Als Steve Jobs 1985 das Unternehmen verließ, gründete er
	kurze Zeit darauf die Firma NeXT, mit welcher er unter anderem das
	Betriebssystem NeXTStep entwickelte, welches auf dem UNIX ähnlichem
	Betriebssystem BSD \cite[S.12]{Tanenbaum2009} und dem
	Mach-2.5-Kernel \cite{MachProject2015} basiert. NeXT wurde 1996 von Apple
	aufgekauft und Jobs kehrte als CEO zu Apple zurück. Damit begann die Karriere
	des mobilen Betriebssystems iOS. Zuerst wurde NeXTStep als Portierung in Form
	von Mac OS X (später OS X) weiter entwickelt. Mac OS X ist wiederrum der
	Vorleger für das iPhone OS (später iOS), welches am 09. Januar 2007 mit dem
	damals neu erschienenen iPhone erstmals vorgestellt wurde.
	\\\\
	\textbf{Apple's Law Enforcement Process Guidelines}
	\\
	Die Firma Apple schreibt auf ihrer Webseite:
	\begin{quote}
		Our commitment to customer privacy doesn't stop because of a government
		information request \cite{AppleGovInfo2015}.
	\end{quote}
	Weiterhin wird beteuert:
	\begin{quote}
		In addition, Apple has never worked with any government agency from any
		country to create a "`back door"' in any of our products or
		services \cite{AppleGovInfo2015}.
	\end{quote}
	Apple beteuert hier, die Privatssphäre des Kunden auch dann zu bewahren,
	wenn die Regierung um Auskunft der Daten bittet.\\
	Zusätzlich wird versichert, dass Apple niemals mit Regierungsbehörden
	jedweder Länder gearbeitet hat, um Trojaner oder andere Hintertüren in eines
	ihrer Produkte oder Dienstleistungen einzubauen.\\
	Grundsätzlich ist eine Einhaltung dieser Versprechen wünschenswert, sowohl für
	Entwickler als auch Endnutzer, da beide Parteien über die proprietäre
	Vorgehensweise, Apple in gewisser Art und Weise vertrauen müssen.
