\section{Härten von iOS}
	Trotz aller vorhandenen Hilfsmittel sollte man nicht auf die Sicherheit von iOS
	vertrauen. Deshalb liste ich abschließend je ein Unterkapitel für Entwickler
	und für Endnutzer, in welchen mögliche Verbesserungen für die
	Entwicklung einer App oder für das Einrichten von iDevices gelistet werden.
	\subsection{Relevantes für den Entwickler}
		Der Entwickler ist verantwortlich für die Sicherheit seiner Anwendung und
		sollte jegliche mögliche Optionen der zusätzlichen Absicherung seiner
		Applikation in Erwägung ziehen. 
		\subsubsection{Passwortstärke}
			Die Verschlüsselung ist eine der wichtigsten Arten seine Daten zu schützen,
			aber auch die kritischste, wenn es um die Implementierung geht. Daher richten
			Hacker ihr Augenmerk zuerst auf die Implementierung und nicht die
			eigentlichen verschlüsselten Daten. Hierbei ist es besonders wichtig, keine
			schwachen Passwörter zu erlauben. Eine strenge Passwortrichtlinie ist
			Pflicht, auch wenn es die Useability negativ beeinflusst. Die gewählten
			Passwörter sollten aus vielen Zeichen bestehen (mindestens 12 Zeichen),
			welche wiederrum zu einem Anteil aus Zahlen, Sonderzeichen und Zeichen, in
			Groß- und Kleinschreibweise, bestehen. Zusätzlich sollten bestimmte Muster,
			wie entlang der QUERTZ-Tastatur zu fahren, einfache Wörter, welche meist in
			Dictionaries enthalten sind und strukturierte Daten wie ein Datum, verhindert
			werden.
		\subsubsection{Common Crypto Library}
			IOS bietet mit der \textsl{Common Crypto Library}\footnote{https://goo.gl/5ApuXL}
			(3CC oder auch CCCrypt) eine Möglichkeit auf C-Ebene 
			Verschlüsselungsalgorithmen wie AES, DES oder 3DES einzusetzen. Dabei bietet
			3CC je nach eingesetztem Algorithmus Block- beziehungsweise Stromchiffre an.
			Zusätzlich wird mit dem \textsl{Cipher Block Chaining} (CBC) eine
			Möglichkeit angeboten, um Man-In-The-Middle, sowie Replay-Angriffe zu
			verhindern. Dies ist möglich, da bei CBC jeder Klartext-Block mit dem
			vorherigen verschlüsselten Chiffre XOR-verknüpft wird und anschließend
			ebenfalls chiffriert wird. Somit ist jeder Block von der bisherigen Ketten
			verschlüsselter Daten abhängig.
		\subsubsection{Sicherung des Hauptschlüssels}			
			Falls ein Master-Key - also ein übergeordneter Schlüssel - zur
			Verschlüsselung eingesetzt wird, sollte dieser zwingend ebenso verschlüsselt
			werden. 
	\subsection{Tips für den Benutzer}
	%TODO