\section{Härten von iOS}
	Trotz aller vorhandenen Hilfsmittel sollte man nicht auf die Sicherheit von iOS
	vertrauen. Deshalb liste ich abschließend je ein Unterkapitel für Entwickler
	und für Endnutzer, in welchen mögliche Verbesserungen für die
	Entwicklung einer App oder für das Einrichten von iDevices gelistet werden.
	\subsection{Relevantes für den Entwickler}
		Der Entwickler ist verantwortlich für die Sicherheit seiner Anwendung und
		sollte jegliche mögliche Optionen der zusätzlichen Absicherung seiner
		Applikation in Erwägung ziehen. 
		\subsubsection{Verschlüsselung implementieren}
			Die Verschlüsselung ist eine der wichtigsten Arten seine Daten zu schützen,
			aber auch die kritischste, wenn es um die Implementierung geht. Daher richten
			Hacker ihr Augenmerk zuerst auf die Implementierung und nicht die
			eigentlichen verschlüsselten Daten. Hierbei ist es besonders wichtig, keine
			schwachen Passwörter zu erlauben. Eine strenge Passwortrichtlinie ist
			Pflicht, auch wenn es die Useability negativ beeinflusst. Die gewählten
			Passwörter sollten aus vielen Zeichen bestehen, welche wiederrum zu einem
			Anteil aus Zahlen, Sonderzeichen und Zeichen, in Groß- und Kleinschreibweise,
			bestehen. Zusätzlich sollten bestimmte Muster verhindert werden, wie
			beispielsweise die  Beispiel über die QUERTZ-Tastatur entlang fahren.
			Passwörter eingeben zu können die
	\subsection{Tips für den Benutzer}
	%TODO