\documentclass{beamer}

\mode<presentation>
{
\usetheme{Warsaw}

\setbeamercovered{transparent}
}
%Deutsche Silbentrennung
\usepackage[ngerman]{babel}
%Deutsche Umlaute
\usepackage[utf8]{inputenc}
%Listen einr�cken
\usepackage{enumitem}
% font definitions, try \usepackage{ae} instead of the following
% three lines if you don't like this look
\usepackage{mathptmx}
\usepackage[scaled=.90]{helvet}
\usepackage{courier}
%Trennung von deutschen Umlauten
\usepackage[T1]{fontenc}


\title{Sicherheit in Android und iOS}

%\subtitle{}

% - Use the \inst{?} command only if the authors have different
%   affiliation.
%\author{F.~Author\inst{1} \and S.~Another\inst{2}}
\author{David Artmann\inst{1} \and Kristoffer Schneider\inst{1}}

% - Use the \inst command only if there are several affiliations.
\institute[Universities of]
{
\inst{1}
Hochschule für angewandte Wissenschaften\\
Würzburg-Schweinfurt
}

\date{\today}


% This is only inserted into the PDF information catalog. Can be left
% out.
\subject{Talks}



% If you have a file called "university-logo-filename.xxx", where xxx
% is a graphic format that can be processed by latex or pdflatex,
% resp., then you can add a logo as follows:
\pgfdeclareimage[height=0.5cm]{university-logo}{media/logo/fhws.png}
\logo{\pgfuseimage{university-logo}}



% Delete this, if you do not want the table of contents to pop up at
% the beginning of each subsection:
\AtBeginSubsection[]
{
\begin{frame}<beamer>
\frametitle{Outline}
\tableofcontents[currentsection,currentsubsection]
\end{frame}
}

% If you wish to uncover everything in a step-wise fashion, uncomment
% the following command:
\beamerdefaultoverlayspecification{<+->}

\begin{document}

% titlepage
\begin{titlepage}
	%Eine mbox wird verwendet um Text zusammenzuhalten
	%vspace erzeugte die in Klammern angegebenen Zeilenabstände
	%baselineskip setzt zeilenabstand
   	\mbox{}\vspace{5\baselineskip}\\
   	%Schriftart und Größe als Attribut
   	\rmfamily\huge
   	%Mittige Textausrichtung (\centerline für eine Zeile)
   	\centering
   	%Das Argument erscheint in Kapitaelchen (small capitals).
	\textsc{Sicherheit in Android und iOS}
	%Umbruch bezogen auf die Hoehe des Kleinbuchstaben x in diesem Element * Faktor
	\\[3ex]
   	Seminararbeit
   	\rmfamily\Large
   	\vspace{1\baselineskip}\\
   	%Externes einbinden einer Textdatei
   	%% versionsnummer entfernt
   	%\input{version.txt}\mbox{}
	\vspace{3\baselineskip}
	Hochschule für angewandte Wissenschaften Würzburg-Schweinfurt
   	\vspace{5\baselineskip}\\
   	\rmfamily\Large
   	David Artmann\\
   	\rmfamily\Large
   	Kristoffer Schneider
   	\vspace{1\baselineskip}\\
   	%Heutiges Datum
   	\today
\end{titlepage}

% toc
\begin{frame}
	\frametitle{Gliederung}
	\tableofcontents
	% You might wish to add the option [pausesections]
\end{frame}


\section{Gegenüberstellung}
	\subsection[Gemeinsamkeiten]{Gemeinsamkeiten}
		%% grober aufbau ist sehr ähnlich, siehe:
		%hardware (TEE, SE)
		%bootvorgang (secure boot chain)
		%userland (sandboxing, rechte)
	\subsection[Unterschiede]{Unterschiede}
		%% unterschiede liegen im detail, siehe:
		%unterschied bei TEE und SE
		%benutzerrechtesystem intern mit linux+unix usern
		%berechtigungen der apps android vs. ios (alles oder nichts)
\section{Systemkultur}
	\subsection{Opensource von Android}
	\subsection{Proprietät unter iOS}
	%% opensource vs. proprietät
	%nachteile: Android: updateproblematik, transparenzprobleme(drittanbieter
	% eigenbauten) vorteile: closed source, proprietär
	%MDM/BYOD: iOS einfacher, Android komplex, schwieriger
	%% Systemmodifikation
	%android erlaubt das rooten, ios nicht (exploits: jailbreak)
	%Feststellen von Rooting unter Android sehr schwer, unter iOS relativ einfach
	% (fork())
	
\section{Härten}
	\subsection{Tips für Endnutzer}
	%% generelle tips
	%% spezifische tips
	\subsection{Ratschläge für Entwickler}
	%% generelle tips
	%
	%% spezifische tips
\begin{frame}
\frametitle{Gemeinsamkeiten}
\framesubtitle{}

\begin{itemize}
  \item test
  \item test
\end{itemize}
\end{frame}

\begin{frame}
\frametitle{}

% You can create overlays
\begin{itemize}
  \item using the \texttt{pause} command:
  \begin{itemize}
    \item First item.
    \pause
    \item Second item.
  \end{itemize}
  \item using overlay specifications:
  \begin{itemize}
    \item<3-> First item.
    \item<4-> Second item.
  \end{itemize}
  \item using the general \texttt{uncover} command:
  \begin{itemize}
    \uncover<5->{\item First item.}
    \uncover<6->{\item Second item.}
  \end{itemize}
\end{itemize}
\end{frame}

\section*{Summary}

\begin{frame}
\frametitle<presentation>{Summary}

\begin{itemize}
  \item The \alert{first main message} of your talk in one or two lines.
\end{itemize}

% The following outlook is optional.
\vskip0pt plus.5fill
\begin{itemize}
  \item Outlook
  \begin{itemize}
    \item Something you haven't solved.
    \item Something else you haven't solved.
  \end{itemize}
\end{itemize}
\end{frame}

\end{document}
