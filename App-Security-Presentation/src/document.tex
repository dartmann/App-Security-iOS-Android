\documentclass{beamer}

\mode<presentation>
{
\usetheme{Warsaw}

\setbeamercovered{transparent}
}
%Deutsche Silbentrennung
\usepackage[ngerman]{babel}
%Deutsche Umlaute
\usepackage[utf8]{inputenc}
%Listen einr�cken
\usepackage{enumitem}
% font definitions, try \usepackage{ae} instead of the following
% three lines if you don't like this look
\usepackage{mathptmx}
\usepackage[scaled=.90]{helvet}
\usepackage{courier}
%Trennung von deutschen Umlauten
\usepackage[T1]{fontenc}


\title{Sicherheit in Android und iOS}

%\subtitle{}

% - Use the \inst{?} command only if the authors have different
%   affiliation.
%\author{F.~Author\inst{1} \and S.~Another\inst{2}}
\author{David Artmann\inst{1} \and Kristoffer Schneider\inst{1}}

% - Use the \inst command only if there are several affiliations.
\institute[Universities of]
{
\inst{1}
Hochschule für angewandte Wissenschaften\\
Würzburg-Schweinfurt
}

\date{\today}


% This is only inserted into the PDF information catalog. Can be left
% out.
\subject{Talks}



% If you have a file called "university-logo-filename.xxx", where xxx
% is a graphic format that can be processed by latex or pdflatex,
% resp., then you can add a logo as follows:
\pgfdeclareimage[height=0.5cm]{university-logo}{media/logo/fhws.png}
\logo{\pgfuseimage{university-logo}}



% Delete this, if you do not want the table of contents to pop up at
% the beginning of each subsection:
\AtBeginSubsection[]
{
\begin{frame}<beamer>
\frametitle{Outline}
\tableofcontents[currentsection,currentsubsection]
\end{frame}
}

% If you wish to uncover everything in a step-wise fashion, uncomment
% the following command:

%\beamerdefaultoverlayspecification{<+->}

\begin{document}

\begin{frame}
	\titlepage
\end{frame}

\begin{frame}
	\frametitle{Outline}
	\tableofcontents
	% You might wish to add the option [pausesections]
\end{frame}


\section{Gegenüberstellung}

\subsection[Short First Subsection Name]{First Subsection Name}

\begin{frame}
\frametitle{}
\framesubtitle{Subtitles are optional}

\begin{itemize}
  \item
  \item
\end{itemize}
\end{frame}

\begin{frame}
\frametitle{}

% You can create overlays
\begin{itemize}
  \item using the \texttt{pause} command:
  \begin{itemize}
    \item First item.
    \pause
    \item Second item.
  \end{itemize}
  \item using overlay specifications:
  \begin{itemize}
    \item<3-> First item.
    \item<4-> Second item.
  \end{itemize}
  \item using the general \texttt{uncover} command:
  \begin{itemize}
    \uncover<5->{\item First item.}
    \uncover<6->{\item Second item.}
  \end{itemize}
\end{itemize}
\end{frame}

\section*{Summary}

\begin{frame}
\frametitle<presentation>{Summary}

\begin{itemize}
  \item The \alert{first main message} of your talk in one or two lines.
\end{itemize}

% The following outlook is optional.
\vskip0pt plus.5fill
\begin{itemize}
  \item Outlook
  \begin{itemize}
    \item Something you haven't solved.
    \item Something else you haven't solved.
  \end{itemize}
\end{itemize}
\end{frame}

\end{document}
